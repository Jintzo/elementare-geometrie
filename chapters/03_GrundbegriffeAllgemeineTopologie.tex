\chapter{Grundbegriffe der allgemeinen Topologie}

\section{Toplogischer Räume}

\begin{definition}{Topologischer Raum}
  Ein \emph{topologischer Raum} ist ein Paar $ (X, \O) $ bestehend aus einer Menge $ X $ und einem System bzw. einer Familie 
  \begin{equation*}
    \O \subseteq \mathcal{P}(X)
  \end{equation*}
  von Teilmengen von $ X $, so dass gilt
  \begin{enumerate}
    \item $ X, \varnothing \in \O $ 
    \item Durchschnitte von \emph{endlich} vielen und Vereinigungen von \emph{beliebig} vielen Mengen aus $ \O $ sind wieder in $ \O $.
  \end{enumerate}
  Ein solches System $ \O $ heißt \emph{Topologie} von $ X $. Die Elemente von $ \O $ heißen \emph{offene Teilmengen} von $ X $. \\
  $ A \subset X $ heißt \emph{abgeschlossen}, falls das Komplement $ X \setminus A $ offen ist.
\end{definition}

\begin{example}{Extrembeispiele}
  \begin{enumerate}
    \item Menge $ X $, \ $ \O_\text{trivial} \coloneqq \{ X, \varnothing \} $ ist die \emph{triviale Topologie}.
    \item Menge $ X $, \ $ \O_\text{diskret} \coloneqq \mathcal{P}(X) $ ist die \emph{diskrete Topologie}.
  \end{enumerate}
\end{example}

\begin{example}{Standard-Topologie auf $ \R $}
  \begin{marginfigure}[4em]
    \textbf{Offenes Intervall}: \\ $ (a,b) \coloneqq \{  t \in \R : a < t < b \} $, \\ $ a $ und $ b $ beliebig
  \end{marginfigure}
  $ X = \R $,
  \begin{equation*}
    \O_{s \text{ (standard)}} \coloneqq \{ I \subset \R : I = \text{Vereinigung von offenen Intervallen} \}
  \end{equation*}
  ist Topologie auf $ \R $.
\end{example}

\clearpage

\begin{example}{Zariski-Topologie auf $ \R $}
  $ X = \R $,
  \begin{equation*}
    \O_\text{Z(ariski)} \coloneqq \{ O \subset \R : O = \R \setminus, \ E \subset \R \text{ endlich} \} \cup \{ \varnothing \}
  \end{equation*}
  ist die \emph{Zariski-Topologie} auf $ \R $. \\
  (Mit anderen Worten: Die abgeschlossenen Mengen sind genau die endlichen Mengen, $ \varnothing $ und $ \R $.) \\
  Diese Topologie spielt eine wichtige Rolle in der algebraischen Geometrie beim Betrachten von Nullstellen von Polynomen: \\
  \begin{align*}
    (a_1 \dots, a_n) &\leftrightarrow p(X) = (X-a_1)\cdots(X-a_n) \\
     \R &\leftrightarrow \text{ Nullpolynom} \\
     \varnothing &\leftrightarrow X^2+1
  \end{align*}
  % TODO Abbildung 2 einfügen
\end{example}

\begin{definition}{Metrischer $ \to $ topologischer Raum}
  Metrische Räume (z.B. $ (X, d) $) sind topologische Räume: \\
  % TODO Abbildung 3
  $ U \subset X $ ist \emph{$ d $-offen} $ \Leftrightarrow \forall p \in U \ \exists \ \epsilon = \epsilon(p) > 0 $, sodass der offene Ball $ B_\epsilon(p) = \{ x \in X : d(x,p) < \epsilon \} $ um $ p $ mit Radius $ \epsilon $ ganz in $ U $ liegt: $ B_\epsilon(p) \subset U $. \\
  Die $ d $-offenen Mengen bilden eine Topologie --- die von der Metrik $ d $ \emph{induzierte Topologie}\sidenote{\textbf{Übungsaufgabe}: Zeigen, dass es sich wirklich um eine Topologie handelt}.
\end{definition}

\begin{definition}{Basis}
  Eine \emph{Basis} für die Topologie $ \O $ ist eine Teilmenge $ \mathcal{B} \subset \O $, sodass für jede offene Menge $ \varnothing \neq V \in \O $ gilt:
  \begin{equation*}
    V = \bigcup_{i \in I}V_i, \quad V_i \in \mathcal{B}\text{.}
  \end{equation*}
  \underline{Beispiel}: $ \mathcal{B} = \{ \text{offene Intervalle} \} $ für Standard-Topologie auf $ \R $.
\end{definition}

\begin{example}{Komplexität einer Topologie}
  $ \R $, $ \C $ haben eine abzählbare Basis bezüglich Standard-Metrik $ d(x,y) = \vert x - y \vert $ (beziehungsweise Standard-Topologie): \\
  Bälle mit rationalen Radien und rationalen Zentren.
  % TODO Abbildung 4
\end{example}

\begin{bla}{Bemerkung --- Gleichheit von Topologien}
  Verschiedene Metriken können die gleiche Topologie induzieren: \\
  Sind $ d, d' $ Metriken auf $ X $ und enthält jeder Ball um $ x \in X $ bezüglich $ d $ einen Ball um $ x $ bezüglich $ d' $ ($ B_{\epsilon'}^d(x) \subset B_\epsilon^d(x) $), dann ist jede $ d $-offene Menge auch $ d' $-offen und somit $ \O(d) \subset \O(d') $. \\
  Gilt auch die Umkehrung ($ \O(d') \subset \O(d) $), so sind die Topologien gleich: $ \O(d) = \O(d') $.
\end{bla}

\begin{example}{Bälle und Würfel sind gleich}
  $ X = \R^2 $, $ x = (x_1, x_2) $, $ y = (y_1, y_2) $
  \begin{align*}
    d(x,y) &\coloneqq \sqrt{(x_1-y_1)^2+(x_2-y_2)^2} \\
    d'(x,y) &\coloneqq \max\{ \vert x_1-y_1\vert, \ \vert x_2-y_2 \vert \}
  \end{align*}
  Die induzierten Topologien sind gleich.
  % TODO Abbildung 5
\end{example}

\begin{example}{Metrische Information sagt nichts über Topologie}
  $ (X, d) $ sei ein beliebiger metrischer Raum,
  \begin{equation*}
    d'(x,y) \coloneqq \frac{d(x,y)}{1 + d(x,y)}
  \end{equation*}
  ist Metrik mit $ \O(d) = \O(d') $. \\
  Für $ d' $ gilt: $ d'(x,y) \leq $ ($ \forall x,y $), insbesondere ist der Durchmesser von $ X $ bezüglich $ d' $:
  \begin{equation*}
    = \sup_{x,y \in X}d'(x,y) \leq 1\text{,}
  \end{equation*}
  das heißt, der Durchmesser eines metrischen Raumes (``metrische Information'') sagt nichts über die Topologie aus.
\end{example}

\begin{definition}{Umgebung}
  % TODO Abbildung 6
  $ (X, \O) $ sei ein topologischer Raum. $ U \subset X $ heißt \emph{Umgebung} von $ A \subset X $, falls
  \begin{equation*}
    \exists \ O \in \O : A \subset O \subset U\text{.}
  \end{equation*}
\end{definition}

\begin{definition}{Innerer Punkt}
  Für $ A \subset X $, $ p \in X $ heißt $ p $ ein \emph{innerer Punkt} von $ A $ (bzw. äußerer Punkt von $ A $), falls $ A $ (bzw. $ X \setminus A $) Umgebung von $ \{ p \} $ ist. \\
  Das \emph{Innere} von $ A $ ist die Menge $ \overset{\circ}{A} $ der inneren Punkte von $ A $.
\end{definition}

\begin{definition}{Abgeschlossene Hülle}
  Die \emph{abgeschlossene Hülle} von $ A $ ist die Menge $ \overline{A} \subset X $, die \underline{nicht} äußere Punkte sind. \\
  \textbf{Beispiel}: $ (a, b) = \{ t \in \R : a < t < b \} $, \\ $ \overline{(a,b)} = [a,b] = \{ t \in \R : a \leq t \leq b \} $.
\end{definition}

\clearpage

\begin{bla}{Drei konstruierte topologische Räume}
  Folgende drei einfache Konstruktionen von neuen topologischen Räumen aus gegebenen:
  \begin{enumerate}

    % -- 1
    \item \textbf{Teilraum-Topologie}: $ (X, \O_X) $ topologischer Raum, $ Y \subseteq X $ Teilmenge.
    \begin{equation*}
      \O_Y \coloneqq \{ U \subseteq Y : \exists \ V \in \O_X \wedge U = V \cap Y \}
    \end{equation*}
    definiert eine Topologie auf $ Y $, die sogenannte \emph{Teilraum-Topologie}.\sidenote{Zu überprüfen!} \\
    \textbf{Achtung!} $ U \in \O_Y $ ist i.a. \underline{nicht} offen in $ X $. Z.B. $ X = \R $, $ Y = [0,1] $, $ V = (-1, 2) $, also $ U = V \cap Y = Y $.
    % TODO Abbildung 1

    % -- 2
    \item \textbf{Produkträume}: $ (X, \O_X) $ und $ (Y, \O_Y) $ zwei topologische Räume. Eine Teilmenge $ W \subseteq X \times Y $ ist \emph{offen} in der \emph{Produkt-Topologie} $ \Leftrightarrow \ \forall (x, y) \in W \ \exists $ Umgebung $ U $ von $ x $ in $ X $ und $ V $ von $ y $ in $ Y $ sodass das ``Kästchen'' $ U \times V \subseteq W $. \\
    \textbf{Achtung!} Nicht jede offene Menge in $ X \times Y $ ist ein Kästchen: die Vereinigung von zwei Kästchen ist beispielsweise auch offen. \\
    \textbf{Beispiel}: $ X = \R $ mit Standard-Topologie, dann ist
    \begin{equation*}
      \underbrace{X \times \cdots \times X}_{x \text{ mal}} = \R^n
    \end{equation*}
    induzierter topologischer Raum.
    % TODO Abbildung 2

    % -- 3
    \item \textbf{Quotienten}: $ (X, \O) $ topologischer Raum, $ \sim $ Äquivalenzrelation\sidenote{Impliziert Partitionierung von $ X $ in disjunkte Teilmengen} auf $ X $. Für $ x \in X $ sei
    \begin{equation*}
      [x] \coloneqq  \{ y \in X : y \sim x \}
    \end{equation*}
    die Äquivalenzklasse von $ x $,
    \begin{equation*}
      X/\sim
    \end{equation*}
    die Menge der Äquivalenzklassen und
    \begin{align*}
      \pi : X &\to X/\sim \\
      x &\mapsto [x]
    \end{align*}
    die kanonische Projektion (surjektiv!). \\
    Die \emph{Quotienten-Topologie} auf $ X/\sim $ nutzt: \\
    $ U \subset X/\sim $ ist \underline{offen} $ \overset{\text{Def.}}{\Leftrightarrow} \pi^{-1}(U) $ ist offen in $ X $. \\
    \textbf{Beispiel}: $ X = \R $ mit Standard-Topologie (induziert durch Standard-Metrik $ d_\R(s,t) = \vert s-t \vert $). \\
    Seien $ s, t \in \R $. Wir definieren
    \begin{equation*}
      s \sim t \overset{\text{Def.}}{\Leftrightarrow} \ \exists \ m \in \Z : t = s + 2\pi m\text{.}
    \end{equation*}
    Dann ist
    \begin{equation*}
      \R/\sim \underset{\text{bijektiv}}{=} S' = \text{ Einheitskreis}\text{.}
    \end{equation*}
    Anstatt dies heuristisch auszudrücken kann dies auch explizit getan werden:
    \begin{align*}
      \R &\to S' = \{ z \in \C : \vert z \vert = 1 \} = \{ (x, y) \in \R : x^2+y^2 = 1 \} \\
      t &\mapsto e^{\i t}\text{.}
    \end{align*}
    % TODO Abbildung 3
    \textbf{Bemerkung}: Andere Interpretation via Gruppen-Aktionen. \\
    $ G = (\Z, +) $ operiert auf $ X = \R $. \\
    \emph{Bahnen-Raum} $ = \R/\sim $ mit
    \begin{align*}
      \Z \times \R &\to \R \\
      (m, t) &\mapsto t + 2\pi m\text{.}
    \end{align*}
    Die Äquivalenzklasse $ [t] $ ist die Bahn von
    \begin{equation*}
      t = \Z*t = \{ t+2\pi m : m \in \Z \}\text{,}
    \end{equation*}
    mehr dazu später.
  \end{enumerate}
\end{bla}

\section{Hausdorffsches Trennungsaxiom}
\begin{bla}{Hausdorffsches Trennungsaxiom $ T_2 $}
  Ein topologischer Raum $ (X,\O) $ heißt \emph{hausdorffsch}, falls man zu je zwei verschiedenen Punkten $ p,q \in X $ disjunkte Umgebungen finden kann, also Umgebungen $ U \ni p $ und $ V \ni q $ mit $ U \cap V = \varnothing $. \\
  \textbf{Beispiel}:
  \begin{enumerate}
    \item Metrische Räume sind hausdorffsch.
    \begin{proof}
       Sei $ d(p,q) \eqqcolon \epsilon $. \\
       Behauptung: $ B_{\epsilon/3}(p) \cap B_{\epsilon/3}(q) = \varnothing $. \\
       Sei $ z $ in $ B_{\epsilon/3}(p) \cap B_{\epsilon/3}(q) $. Dann gilt
       \begin{equation*}
         d(p,q) \overset{\triangle\text{-Ugl.}}{\leq} d(p,z)+d(z,q) \leq \tfrac{\epsilon}{3} + \tfrac{\epsilon}{3} + \tfrac{2\epsilon}{3} > \epsilon \quad \lightning
       \end{equation*}
     \end{proof} 
    \item $ (\R, \O_\text{standard}) $ ist hausdorffsch, da die Standard-Topologie von der Metrik induziert wird.
    \item $ (\R, \O_\text{Zariski}) $ ist \underline{nicht} hausdorffsch: offene Mengen sind Komplemente von endlich vielen Punkten, also für $ p, q \in \R, \ p \neq q $:
    \begin{align*}
      U_p &= \R \setminus \{ p_1, \dots, p_n \} \\
      U_q &= \R \setminus \{ q_1, \dots, q_k \}\text{,}
    \end{align*}
    also $ U_p \cap U_q \neq \varnothing $.
  \end{enumerate}
  \begin{marginfigure}
    \textbf{Erinnerung --- Konvergenz}. \\
    $ (x_n)_{n \in \N} \subset X $ (top. Raum). $ X \ni a $ heißt \emph{Limes} um $ (x_n)_{n \in \N} $ falls es zu jeder Umgebung $ U $ von $ a $ ein $ n_0 \in \N $ gibt, sodass $ x_n \in U \ \forall n \geq n_0 $.
  \end{marginfigure}
  \textbf{Wichtige Konsequenz von ``hausdorffsch''}: In einem Hd-Raum hat jede Folge höchstens einen Limespunkt/Grenzwert.
\end{bla}

\begin{bla}{Bemerkung}
  \begin{enumerate}
    \item Jeder Teilraum (mit TR-Topologie) eines Hd-Raumes ist Hd. 
    \item $ X, Y $ Hd-Räume $ \Rightarrow $ $ X \times Y $ ist Hd-Raum bezüglich Produkt-Topologie.
  \end{enumerate}
\end{bla}

\section{Stetigkeit}

\begin{definition}{Stetigkeit}
  $ (X, \O_X) $, $ (Y, \O_Y) $ topologische Räume. Eine Abbildung $ f : X \to Y $ heißt \emph{stetig}, falls die Urbilder von offenen Mengen in $ Y $ offen sind in $ X $.
  % TODO Abbildung 4
\end{definition}

\begin{example}{Einfache Stetigkeiten}
  \begin{enumerate}
    \item $ \text{Id}: X \to X $, $ x \mapsto x $ ist stetig.
    \item Die Komposition von stetigen Abbildungen ist stetig.
    \item Für $ (X, \O) = (\R, \O_\text{standard}) = (Y, \O_Y) $ gibt es unendlich viele Beispiele in Analysis I. \\
    Für metrische Räume ist diese  Definition äquivalent zur $ \epsilon $-$ \delta $-Definition und zur Folgenstetigkeit\sidenote{Übungsaufgabe!}.
  \end{enumerate}
\end{example}

\begin{definition}{Homöomorphismus}
  \begin{itemize}
    \item Eine bijektive Abbildung $ f: X \to Y $ zwischen topologischen Räumen heißt \emph{Homöomorphismus}, falls $ f $ und $ f^{-1} $ stetig sind.
    \item $ X $ und $ Y $ heißen \emph{homöomorph}, falls ein Homöomorphismus $ f: X \to Y $ existiert (notiere $ X \cong Y $). 
  \end{itemize}
\end{definition}

\begin{bla}{Bemerkung --- Homöomorphismengruppe}
  \begin{itemize}
    \item $ \text{Id}_X: X \to X $, $ x \mapsto x $ ist Homöomorphismus.
    \item Verkettungen von Homöomorphismen sind wieder Homöomorphismen.
    \item Inverses eines Homöomorphismus ist ein Homöomorphismus. \\
      Aus diesen drei Punkten folgt, dass die Homöomorphismen eine Gruppe bilden.
  \end{itemize}
\end{bla}

\begin{example}{Einfache Homöomorphismen}
  \begin{itemize}
    \item $ [0,1] = \{ t \in \R : 0 \leq t \leq 1 \} \cong [a,b] $ mit $ a < b \in \R $ \\ (via $ f(t) = a+t(b-a) $).
    \item $ (0,1) = \{ t \in \R : 0 < t < 1 \} \cong (a,b) $ mit $ a < b $ beliebig. 
    \item $ \R \cong (-1, 1) \cong (0,1) $ \\ (z.B. via $ t \mapsto \tanh t = \tfrac{e^{2t}-1}{e^{2t}+1} $).
    \item Stetig und injektiv, aber \underline{kein} Homöomorphismus! \\
      $ f: [0,1) \to S^1 $, $ t \mapsto e^{2\pi\i t} = \cos(2\pi t) + \i\sin(2\pi t) $ ist stetig, injektiv, aber kein Homöomorphismus.
    \item Projektions-Abbildungen sind stetig, z.B. $ p_1: X_1 \times X_2 \to X_1 $, $ (x_1,x_2) \mapsto x_1 $: Für $ U $ offen in $ X_1 $ ist $ p^{-1}(U) = U \times X_2 $ offen bezüglich der Produkttopologie.
    \item Metrische Räume $ (X, d_X) $, $ (Y, d_Y) $ und Isometrie $ f: X \to Y $, also eine bijektive Abbildung, so dass
      \begin{equation*}
        \forall x, y \in X : d_Y(f(x), f(y)) = d_X(x, y)\text{.}
      \end{equation*}
      \textbf{Behauptung}: $ f $ ist Homöomorphismus (bzgl. der durch Metriken definierten Topologien). \\
      \textbf{Beweis} (über $ \epsilon $-$ \delta $-Definition): $ \delta \coloneqq \epsilon $. \\
        $ d_X(x, y) < \delta \Rightarrow d_Y(f(x), f(y)) = d_X(x,y) < \delta = \epsilon $, also ist $ f $ stetig. \\
        Analog für $ f^{-1} $.
    \item $ S^n = \{ x \in R^{n+1} : \Vert x \Vert^2 = 1 \} $ ist die $ n $-dimensionale Einheitssphäre in $ \R^{n+1} $. \\
      $ e_{n+1} = (0,\dots,0,1) $ sei der ``Nordpol'' von $ S_n $. \\
      \textbf{Behauptung}: $ S^n\setminus\{ e_{n+1} \} \cong \R^n $. \\
      \textbf{Beweis} (via stereographische Projektion):
      \begin{equation*}
        \R^n \cong \{ x \in \R^{n+1} : x_{n+1} = 0 \}\text{,}
      \end{equation*}
      \begin{equation*}
        f(x) \coloneqq (\tfrac{x_1}{1-x_{n+1}}, \dots, \tfrac{x_n}{1-x_{n+1}}) \text{ stetig,}
      \end{equation*}
      \begin{equation*}
        f^{-1}: \R^n \to S^n\text{,} \quad y \mapsto \left( \tfrac{2y_1}{\Vert y \Vert^2+1}, \dots, \tfrac{2y_n}{\Vert y \Vert^2+1}, \tfrac{\Vert y \Vert^2-1}{\Vert y \Vert^2+1} \right)\text{ auch stetig.}
      \end{equation*}
      Also ist $ f $ homöomorph. \\
      \textbf{Achtung}: $ S^n $ ist \underline{nicht} homöomorph zu $ \R^n $ (da $ S^n $ kompakt und $ \R^n $ nicht kompakt ist, mehr dazu später).
  \end{itemize}
\end{example}

\begin{bla}{Bemerkung --- Isometrien-Untergruppe}
  Isometrien bilden eine Untergruppe der Homöomorphismen von $ X $ (versehen mit von der Metrik induzierten Topologie):
  \begin{equation*}
    \text{Isom}(X, d) \subseteq \text{Homö}(X, \O_d) \subseteq \text{Bij}(X)\text{.}
  \end{equation*}
\end{bla}

\begin{bla}{Exkurs 1 --- Kurven}
  Was ist eine Kurve? \\
  \emph{Naive Definition}: Eine Kurve ist ein stetiges Bild eines Intervalls. \\
  \textbf{Problem}: $ \exists $ stetige, surjektive (aber nicht injektive) Abbildungen $ I = [0,1] \to I^2 $ (``Peano-Kurven'', ``space-filling curves'')\sidenote{Mehr dazu in Königsberger --- Analysis I.}. \\
  \textbf{Ausweg 1}: \emph{Jordan-Kuven} (bzw. geschlossene J-Kurven). \\
    $ \coloneqq $ top. Raum, homöomorph zu $ I = [0,1] $ (J-Kurve) \\
    $ \coloneqq $ top. Raum, homöomorph zu $ S^1 $ (geschlossene J-Kurve) \\
  \textbf{Ausweg 2}: \emph{reguläre stetig differenzierbare Kurven} (lokal injektiv). \\
  \textbf{Verwendung}: z.B. \emph{Knoten} --- spezielle geschlossene Jordankurve als Unterraum von $ \R^3 $:
  \begin{equation*}
    \exists \ f : S^1 \to \R^3 \text{ mit } f(S^1) \cong S^1
  \end{equation*}
  mit Teilraumtopologie von $ R^3 $. \\
  Zwei Knoten $ K_1 $, $ K_2 \subset \R^3 $ sind \emph{äquivalent}, falls es einen Homöomorphismus $ h $ von $ \R^3 $ gibt mit $ h(K_1) = K_2 $.\sidenote{\textbf{Knotentheorie} studiert die Äquivalenz von Knoten, siehe z.B. Sossinsky --- Mathematik der Knoten}
\end{bla}

\begin{bla}{Exkurs 2 --- Topologische Gruppen}
  Eine topologische Gruppe ist eine Gruppe versehen mit einer Topologie, sodass die Gruppenmultiplikation
  \begin{equation*}
    m: G \times G \to G, \quad (g,h) \mapsto g*h
  \end{equation*}
  mit Produkt-Topologie und die Inversenbildung
  \begin{equation*}
    i: G \to G, \quad g \mapsto g^{-1 }
  \end{equation*}
  stetig sind.
\end{bla}

\begin{example}{Topologische Gruppen}
  \begin{enumerate}
    \item $ G $ beliebige Gruppe mit diskreter Topologie ist topologische Gruppe.
    \item $ \R^n $ mit Standard-Topologie ist abelsche topologische Gruppe.
    \item $ \R \setminus \{ 0 \} $, $ \C \setminus \{ 0 \} $ sind multiplikative topologische Gruppen.
    \item $ H \subset G $ Untergruppe einer topologischen Gruppe ist topologische Gruppe bzgl. Teilraumtopologie.
    \item Das Produkt von topologischen Gruppen mit Produkttopologie ist eine topologische Gruppe.
    \item $ \text{GL}(n,\R) = \{ A \in \underbrace{\R^{n \times n}}_{= \R^{n^2}} : \det A \neq 0 \} $ allg. reelle lineare Gruppe. \\
      $ \text{GL}(n,\R) \subset \R^{n^2} $ versehen mit Teilraum-Topologie induziert von $ \R^{n^2} = \R^{n \times n} $ ist topologische Gruppe:
      \begin{itemize}
        \item Matrizenmultiplikation ist stetige Abbildung ($ \R^{n^2} \times \R^{n^2} \to \R^{n^2} $),
        \item Inversen-Abbildung ist ebenfalls stetig (wegen expliziter Formel für $ A^{-1} $).
      \end{itemize}
    \item $ \text{SO}(n) = \{ A \in \text{GL}(n,\R) : A^\top A = E_n, \det A = 1 \} $ ist die \emph{spezielle orthogonale Gruppe}. Sie ist eine topologische Gruppe nach Beispiel 4 und 6. \\
      Insbesondere ist
      \begin{equation*}
        \text{SO}(2) = \left\{ \begin{pmatrix}
          \cos \theta & -\sin \theta \\
          \sin \theta & \cos \theta
        \end{pmatrix} : \theta \in [0, 2\pi] \right\} \cong S'
      \end{equation*}
      eine abelsche topologische Gruppe.
  \end{enumerate}
\end{example}

\section{Zusammenhang}

\begin{definition}{Zusammenhängend}
  Ein topologischer Raum $ (X, \O) $ heißt \emph{zusammenhängend}, falls $ \varnothing $ und $ X $ die einzigen gleichzeitig offenen und abgeschlossenenen Teilmengen von $ X $ sind. \\
  \textbf{Äquivalent}: $ X $ ist zusammenhängend $ \Leftrightarrow X $ ist \emph{nicht} disjunkte Vereinigung von 2 offenen, nichtleeren Teilmengen. \\
  \emph{Beweis}: $ A \subset X $ offen und abgeschlossen $ \Leftrightarrow A $ und $ X \setminus A $ offen $ \Leftrightarrow A $ und $ X \setminus A $ abgeschlossen.
\end{definition}

\begin{example}{Zusammenhang}
  \begin{enumerate}
    \item $ \R $ (und ebenso beliebige Intervalle) ist zusammenhängend, $ \R \setminus \{ 0 \} $ ist \emph{nicht} zusammenhängend. \\
      \textbf{Beweis}: Sei $ I \subseteq \R $ (abgeschlossenes oder offenes oder halboffenes) Intervall. \\
      \emph{Annahme}: $ I \neq U \neq \varnothing $, sei eine offen-abgeschlossene Teilmenge von $ I $. Dann gibt es mindestens einen Punkt $ u \in U $ und $ v \in I \setminus U $. OBdA $ u < v $. Setze $ U_0 \coloneqq \{ x \in U : x < v \} $ und $ c \coloneqq \sup U_0 $. Also $ u \leq c \leq v $. Weiter ist $ c \in U $, da $ U $ abgeschlossen ist. Eine ganze Umgebung von $ c $ gehört auch zu $ U $, da $ U $ offen ist. Damit gehört eine ganze Umgebung von $ c $ auch zu $ U_0 \quad \lightning $
  \end{enumerate}
\end{example}

\begin{bla}{Ergänzung --- Zusammenhang von Teilmengen}
  Allgemein: Eine \emph{Teilmenge} $ B \subset X $ heißt \emph{zusammenhängend}, falls sie bezüglich der Teilraumtopologie zusammenhängend ist.
\end{bla}

\begin{bla}{Bemerkung --- Einpunktige Mengen}
  Einpunktige Mengen sind zusammenhängend: $ \{ x \} $ mit Teilraumtopologie ist diskret (also sind $ \{ x \} $ und $ \varnothing $ die einzigen offenen Mengen).
\end{bla}

\begin{definition}{Zusammenhangskomponente}
  Sei $ x \in X $. Die \emph{Zusammenhangskomponente} $ Z(x) $ ist die Vereinigung aller zusammenhängenden Teilmengen, die $ x $ enthalten.
\end{definition}

\begin{lemma}{Eigenschaften zusammenhängender Mengen}
  \begin{enumerate}
    \item $ A $ ist zusammenhängend $ \Rightarrow \overline{A} $ (abgeschlossene Hülle von $ A $) ist zusammenhängend.
    \item $ A $, $ B $ zusammenhängend, $ A \cap B \neq \varnothing \Rightarrow A \cup B $ zusammenhängend.\sidenote{Übungsaufgabe, es wird nur die Definition von Zusammenhang benötigt.} 
  \end{enumerate}
\end{lemma}