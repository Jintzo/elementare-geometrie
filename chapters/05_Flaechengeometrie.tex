\chapter{Geometrie von Flächen}

Ziel dieses Kapitels ist es, auf zweidimensionalen Mannigfaltigkeiten (beziehungsweise Flächen im $ \R^3 $) ``Geometrie'' zu betreiben  (also beispielsweise Längen und Winkel messen und so weiter).\footnote{Für mehr Informationen: \textbf{Gauß} (1827) vgl. \textbf{Spirak}: \emph{A comprehensive introduction to differential geometry} Vol. III --- how to read Gauß}

\section{Reguläre Flächen in $ \R^3 $}

\begin{remark}[Erinnerung an reguläre Flächen]
  Eine Teilmenge $ S \subset \R^3 $ ist eine \term{reguläre Fläche}, falls es zu jedem Punkt $ p \in S $ eine offene Umgebung $ V $ von $ p $ in $ \R^3 $, eine offene Teilmenge $ U \subset \R^2 $ und eine $ C^\infty $-Abbildung
  \begin{equation*}
    x: U \ni (u,v) \mapsto x(u,v) = \left( x_1(u,v),x_2(u,v),x_3(u,v) \right) \in \R^3
  \end{equation*}
  gibt mit:
  \begin{enumerate}
    \item $ x(U) = S \cap V $ und $ x: U \to S \cap V $ ist ein Homöomorphismus. 
    \item Das Differenzial $ \text{d}x|_{(u,v)}: \underset{\cong \R^2}{T_{(u,v)}\R^2} \to \underset{\cong \R^3}{T_{x(u,v)}\R^3} $ ist injektiv ($ \forall (u,v) \in U $)
    \begin{itemize}
      \item[$ \Leftrightarrow $] Die Funktionalmatrix
        \begin{equation*}
          \begin{pmatrix}
            \frac{\partial x_1}{\partial u}(u,v) & \frac{\partial x_1}{\partial v}(u,v) \\
            \frac{\partial x_2}{\partial u}(u,v) & \frac{\partial x_2}{\partial v}(u,v) \\
            \frac{\partial x_3}{\partial u}(u,v) & \frac{\partial x_3}{\partial v}(u,v)
          \end{pmatrix} \eqqcolon \begin{pmatrix}
            x_u(u,v) & x_v(u,v)
          \end{pmatrix}
        \end{equation*}
        hat Rang $ 2 $ ($ \forall (u,v) \in U $).
      \item[$ \Leftrightarrow $] $ x_u(u,v) $, $ x_v(u,v) $ sind linear unabhängig ($ \forall (u,v) \in U $).
      \item[$ \Leftrightarrow $] Vektorprodukt $ x_u(u,v) \times x_v(u,v) \neq 0 $ ($ \forall (u,v) \in U $).
    \end{itemize}
  \end{enumerate}
\end{remark}

\begin{remark}[Erinnerung an das Kreuz-/Vektorprodukt]
  \ \\
  $ a \coloneqq (a_1, a_2, a_3) \in \R^3 $ und $ b \coloneqq (b_1, b_2, b_3) \in \R^3 $.
  \begin{equation*}
    a \wedge b \ (\cong a \times b) \ \coloneqq (a_2b_3-a_3b_2, a_3b_1-a_1b_3, a_1b_2-a_2b_1) \in \R^3\text{.}
  \end{equation*}
  \emph{Eigenschaften}:
  \begin{enumerate}
    \item $ a \wedge b \bot a $ \quad $ a \wedge b \bot b $ 
    \item $ \det(a, b, a \wedge b) \geq 0 $
    \item $ a \wedge (-b) = - (a \wedge b) $
    \item $ \Vert a \wedge b \Vert = \Vert a \Vert * \Vert b \Vert * \sin \alpha $ (Winkel zwischen den Vektoren), Fläche des von $ a $ und $ b $ aufgespannten Parallelogramms
  \end{enumerate}
\end{remark}

\begin{definition}[Tangentialraum]
  Der \term{Tangentialraum}\label{def:tangentialraum} in Punkt $ p \in \R^3 $ ist der affine Unterraum
  \begin{equation*}
    T_p\R^3 \coloneqq \{ p \} \times \R^3 = \{ (p,v) : v \in \R^3 \}\text{.}
  \end{equation*}
  Für eine reguläre Fläche $ S $ und $ p = x(u,v) \in S $ ist die \term{Tangentialebene}\label{def:tangentialebene} in $ p \in S $ definiert als
  \begin{equation*}
    T_pS \coloneqq \text{d}x_{(u,v)}\left(T_{(u,v)}\R^2\right) \coloneqq \{ p \} \times [x_u(u,v), x_v(u,v)] \subset T_p\R^3
  \end{equation*}
  $ 2 $-dimensionaler, affiner Unterraum des $ \R^3 $.
\end{definition}