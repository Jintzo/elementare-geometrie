\chapter{Geometrie von Flächen}

Ziel dieses Kapitels ist es, auf zweidimensionalen Mannigfaltigkeiten (beziehungsweise Flächen im $ \R^3 $) ``Geometrie'' zu betreiben  (also beispielsweise Längen und Winkel messen und so weiter).\footnote{Für mehr Informationen: \textbf{Gauß} (1827) vgl. \textbf{Spirak}: \emph{A comprehensive introduction to differential geometry} Vol. III --- how to read Gauß}

\section{Reguläre Flächen in $ \R^3 $}

\begin{remark}[Erinnerung an reguläre Flächen]
  Eine Teilmenge $ S \subset \R^3 $ ist eine \term{reguläre Fläche}, falls es zu jedem Punkt $ p \in S $ eine offene Umgebung $ V $ von $ p $ in $ \R^3 $, eine offene Teilmenge $ U \subset \R^2 $ und eine $ C^\infty $-Abbildung
  \begin{equation*}
    x: U \ni (u,v) \mapsto x(u,v) = \left( x_1(u,v),x_2(u,v),x_3(u,v) \right) \in \R^3
  \end{equation*}
  gibt mit:
  \begin{enumerate}
    \item $ x(U) = S \cap V $ und $ x: U \to S \cap V $ ist ein Homöomorphismus. 
    \item Das Differenzial $ \text{d}x|_{(u,v)}: \underset{\cong \R^2}{T_{(u,v)}\R^2} \to \underset{\cong \R^3}{T_{x(u,v)}\R^3} $ ist injektiv ($ \forall (u,v) \in U $)
    \begin{itemize}
      \item[$ \Leftrightarrow $] Die Funktionalmatrix
        \begin{equation*}
          \begin{pmatrix}
            \frac{\partial x_1}{\partial u}(u,v) & \frac{\partial x_1}{\partial v}(u,v) \\
            \frac{\partial x_2}{\partial u}(u,v) & \frac{\partial x_2}{\partial v}(u,v) \\
            \frac{\partial x_3}{\partial u}(u,v) & \frac{\partial x_3}{\partial v}(u,v)
          \end{pmatrix} \eqqcolon \begin{pmatrix}
            x_u(u,v) & x_v(u,v)
          \end{pmatrix}
        \end{equation*}
        hat Rang $ 2 $ ($ \forall (u,v) \in U $).
      \item[$ \Leftrightarrow $] $ x_u(u,v) $, $ x_v(u,v) $ sind linear unabhängig ($ \forall (u,v) \in U $).
      \item[$ \Leftrightarrow $] Vektorprodukt $ x_u(u,v) \times x_v(u,v) \neq 0 $ ($ \forall (u,v) \in U $).
    \end{itemize}
  \end{enumerate}
\end{remark}

\begin{remark}[Erinnerung an das Kreuz-/Vektorprodukt]
  \ \\
  $ a \coloneqq (a_1, a_2, a_3) \in \R^3 $ und $ b \coloneqq (b_1, b_2, b_3) \in \R^3 $.
  \begin{equation*}
    a \wedge b \ (\cong a \times b) \ \coloneqq (a_2b_3-a_3b_2, a_3b_1-a_1b_3, a_1b_2-a_2b_1) \in \R^3\text{.}
  \end{equation*}
  \emph{Eigenschaften}:
  \begin{enumerate}
    \item $ a \wedge b \bot a $ \quad $ a \wedge b \bot b $ 
    \item $ \det(a, b, a \wedge b) \geq 0 $
    \item $ a \wedge (-b) = - (a \wedge b) $
    \item $ \Vert a \wedge b \Vert = \Vert a \Vert * \Vert b \Vert * \sin \alpha $ (Winkel zwischen den Vektoren), Fläche des von $ a $ und $ b $ aufgespannten Parallelogramms
  \end{enumerate}
\end{remark}

\begin{definition}[Tangentialraum]
  Der \term{Tangentialraum}\label{def:tangentialraum} in Punkt $ p \in \R^3 $ ist der affine Unterraum
  \begin{equation*}
    T_p\R^3 \coloneqq \{ p \} \times \R^3 = \{ (p,v) : v \in \R^3 \}\text{.}
  \end{equation*}
  Für eine reguläre Fläche $ S $ und $ p = x(u,v) \in S $ ist die \term{Tangentialebene}\label{def:tangentialebene} in $ p \in S $ definiert als
  \begin{equation*}
    T_pS \coloneqq \text{d}x_{(u,v)}\left(T_{(u,v)}\R^2\right) \coloneqq \{ p \} \times [x_u(u,v), x_v(u,v)] \subset T_p\R^3
  \end{equation*}
  $ 2 $-dimensionaler, affiner Unterraum des $ \R^3 $.
\end{definition}

\begin{remark}[Geometrische Interpretation des Tangentialraums]
  \
  % TODO Grafik einfügen
  \begin{equation*}
    x_u(u_0,v_0) = \frac{\partial x}{\partial u}(u_0,v_0) = \frac{\text{d}}{\text{d}t}|_{t = 0}x(u_0+t,v_0) = \text{d}x|_{(u_0,v_0)}(e_1)
  \end{equation*}
  \emph{Allgemein}: \\*
  Sei $ c(t) \coloneqq x(u(t),v(t)) $ eine \term{Flächenkurve}\label{def:flaechenkurve} in $ x(U) $ durch den Punkt $ x(u(0),v(0)) = x(u_0,v_0) $. \\*
  \term{Tangentialvektor}\label{def:tangentialvektor} an $ c $ im Punkt $ x(u_0,v_0): $
  \begin{align*}
    c'(0) &= \frac{\text{d}c}{\text{d}t}|_{t=0} = \frac{\text{d}}{\text{d}t}x(u(t),v(t))|_{t = 0} \\
      &= \frac{\partial x}{\partial u}(u(0),v(0))u'(0) + \frac{\partial x}{\partial v}v'(0) = x_u(u_0,v_0)u'(0)+x_v(u_0,v_0)v'(0)
  \end{align*}
  \emph{Also}: Tangentialebene in $ x(u_0,v_0) = $ Menge aller Tangentialvektoren als Flächenkurven.
\end{remark}

\begin{remark}[Parameterisierungsunabhängigkeit obiger Definitionen]
  \ \\
  Sei $ \overline{x} : \overline{U} \to \overline{x}(\overline{U}) = x(U) $ eine andere Parametrisierung von $ S $ um $ p = x(u_0, v_0) = \overline{x}(\overline{u_0}, \overline{v_0}) $. \\
  \emph{Zu zeigen}: Die lineare Hüllen sind gleich: $ [\overline{x}_{\overline{u}}, \overline{x}_{\overline{v}}] = [x_u, x_v] $. \\
  Es ist $ k \coloneqq \overline{x}^{-1}*x : U \to \overline{U} $ die Koordinatentransformation:
  \begin{equation*}
    x_u = \frac{\partial x}{\partial u}(u,v) = \frac{\partial \overline{x}(\overline{x}^{-1} \circ x)}{\partial u}(u,v) = \frac{\partial \overline{x}}{\partial u}(\overline{u}, \overline{v}) = \frac{\partial \overline{x}}{\partial \overline{u}}(\overline{u},\overline{v}) *\frac{\partial \overline{u}}{\partial u}(u,v)+\frac{\partial \overline{x}}{\partial \overline{v}} * \frac{\partial \overline{v}}{\partial u}
  \end{equation*}
  Entsprechend:
  \begin{align*}
    x_u &= \overline{x}_{\overline{u}}\frac{\partial \overline{u}}{\partial u} + \overline{x}_{\overline{v}}\frac{\partial \overline{v}}{\partial u} \quad \text{d.h. $ x_u $ ist Linearkombination von $ \overline{x}_{\overline{u}} $ und $ \overline{x}_{\overline{v}} $} \\
    x_v &= \overline{x}_{\overline{u}}\frac{\partial \overline{u}}{\partial v} + \overline{x}_{\overline{v}}\frac{\partial \overline{v}}{\partial v} \quad \text{d.h. $ x_v $ ist Linearkombination von $ \overline{x}_{\overline{u}} $ und $ \overline{x}_{\overline{v}} $}
  \end{align*}
  Also $ [x_u, x_v] = [\overline{x}_{\overline{u}}, \overline{x}_{\overline{v}}] $, verschiedene Basen von $ T_pS $ mit Basis-Transformations-Matrix
  \begin{equation*}
    D(u,v) = \begin{pmatrix}
      \frac{\partial \overline{u}}{\partial u} & \frac{\partial \overline{u}}{\partial v} \\
      \frac{\partial \overline{v}}{\partial u} & \frac{\partial \overline{v}}{\partial v}
    \end{pmatrix}\text{.}
  \end{equation*}
  Das ist die Funktionalmatrix der Parametertransformation. Insb. ist $ \det D(u,v) \neq 0 $.
\end{remark}

\begin{example}
  \
  \begin{enumerate}
    \item \emph{affine Ebene}: $ a_0, a, b \in \R^3 $, $ S \coloneqq \{ a_0 + ua + vb : u,v \in \R \} $ ist reguläre Fläche, falls $ a $ und $ b $ linear unabhängig sind. Mit
    \begin{equation*}
      U \coloneqq \R^2, \ V \coloneqq \R^3, \ x: \R^2 \ni (u,v) \mapsto a_0 + ua + vb \in \R^3\text{,}
    \end{equation*}
    \begin{equation*}
      x_u  =\frac{\partial x}{\partial u} = a, \ x_v = \frac{\partial x}{\partial v} = b, \ T_{x(u,v)}S = \{ x(u,v) \} \times [a,b] \cong S\text{.}
    \end{equation*}

    \item $ U \subseteq \R^2 $, $ f: U \to \R $ $ C^\infty $-Funktion, $ S \coloneqq $ Graph von $ f \coloneqq \{ (x_1, x_2, x_3) \in \R^3 : (x_1, x_2) \in U, \ x_3 = f(x_1, x_2) \} $. \\
    \emph{Behauptung}: $ S $ ist reguläre Fläche. \\
    $ U = U $, $ V = \R^3 $, $ x : U \ni (u,v) \mapsto (u,v, f(u,v)) \in \R^3 $. \\
    $ x(U) = S = S \cap V $, $ x: U \to S $ stetig und $ x^{-1}: S \ni (u,v,f(u,v)) \mapsto (u,v) \in U $ ist als Projektion auch stetig. Also ist $ x $ ein Homöomorphismus. \\
    Weiter ist
    \begin{align*}
      x_u &= \left(1,0, \frac{\partial f}{\partial u}\right), \\
      x_v &= \left( 0,1, \frac{\partial f}{\partial v} \right)
    \end{align*}
    also sind $ x_u $ und $ x_v $ linear unabhängig.
  \end{enumerate}
\end{example}

\begin{remark}
  Ist $ S $ reguläre Fläche in $ \R^3 $, so existiert zu jedem Punkt $ p \in S $ eine offene Umgebung $ O \subset \R^3 $, so dass $ S \cap O $ Graph einer $ C^\infty $-Funktion ist (beispielsweise $ S^2 = $ 2-Sphäre vom Radius $ 1 $).
\end{remark}

\section{Erste Fundamentalform einer regulären Fläche}

\begin{remark}[Erinnerung an LA]
  Modell der euklidischen Geometrie: \\*
  $ \R $-Vektorraum + Skalarprodukt = euklidischer $ (V, \langle \cdot, \cdot \rangle) $-Vektorraum \\*
  $ \leadsto \Vert a \Vert = \sqrt{\langle a, a \rangle} $ \textbf{Länge} eines Vektors $ a \in V $ \\*
  $ \leadsto \cos \measuredangle(a,b) = \frac{\langle a, b \rangle}{\Vert a \Vert*\Vert b \Vert} = \left\langle \frac{a}{\Vert a \Vert}, \frac{b}{\Vert b \Vert} \right\rangle $ \textbf{Winkel} \\
\end{remark}

\begin{remark}[Übertragung auf gekrümmte Flächen (Gauß)]
  Sei $ S $ eine reguläre Fläche und $ p \in S $. Betrachte die bilineare Abbildung
  \begin{equation*}
    \langle \cdot, \cdot \rangle_p : T_pS \times T_pS \to T_pS\text{,} \quad \langle a, b \rangle_p \coloneqq \langle a, b \rangle
  \end{equation*}
  (identifiziere affine Ebene mit VR $ R^2 $, $ \langle a, b \rangle $ ist Standard-SKP in $ \R^3 $). \\
  Die Zuordnung $ I: p \mapsto I_p \coloneqq \langle \cdot, \cdot \rangle_p $ heißt \term{1. Fundamentalform der Fläche $ S $}\label{def:ersteFundamentalform}. \\
  Ist $ x: U \ni (u,v) \mapsto x(u,v) \in S $ eine lokale Parametrisierung von $ S $ (um $ p \in S $), so bilden $ x_u(u,v) $ und $ x_v(u,v) $ eine Basis von $ T_{x(u,v)}S $. Bezüglich dieser können wir $ I_p $, $ p \in x(U) \subset S $ durch eine positiv definite symmetrische $ 2 \times 2 $-Matrix darstellen:
  \begin{equation*}
      \left( \underbrace{g_{ij}(u,v)}_{\in \R^{2 \times 2}} \right) = \begin{pmatrix}
        g_{11}(u,v) & g_{12}(u,v) \\
        g_{21}(u,v) & g_{22}(u,v)
      \end{pmatrix} = \underset{\text{Originalnotation Gauß}}{\begin{pmatrix}
        E(u,v) & F(u,v) \\
        F(u,v) & G(u,v)
      \end{pmatrix}}
  \end{equation*}
  mit
  \begin{align*}
    g_{11}(u,v) = E(u,v) &= \langle x_u(u,v),x_u(u,v) \rangle_p = \underset{\text{Standard-SKP von }\R^3}{\langle x_u(u,v),x_u(u,v) \rangle} \\
    g_{12}(u,v) &= \langle x_u, x_v \rangle_p = \langle x_v, x_u \rangle = g_{21}(u,v) \\
    g_{22}(u,v) &= \langle x_v, x_v \rangle_p = \langle x_v, x_v \rangle
  \end{align*}
  insbesondere sind die $ g_{ij}: U \to \R $ $ C^\infty $-Funktionen. \\
  \emph{Also}: $ \left( g_{ij}(u,v) \right) $ ist eine Familie $ \in \R^{2\times 2} $ von Skalarprodukten, die differenzierbar von $ (u,v) $ abhängig ist. (Riemannsche Metrik)
\end{remark}

\begin{remark}[Bedingungen an obige Matrix]
  \
  \begin{enumerate}
    \item \emph{Hurwitz}: $ I_p $ ist positiv definit $ \Leftrightarrow E = g_{11} > 0 $, $ E*G-F^2 = \det(g_{ij}) > 0 $.
    \item \emph{andere Parametrisierung}: $ \overline{x}(\overline{u},\overline{v}) \leadsto $ neue Basis $ \{ \overline{x}_{\overline{u}}, \overline{x}_{\overline{v}} \} $, Matrix von $ I $ bezüglich dieser Basis $ \left( \underset{\in \R^{2\times 2}{\overline{g}_{ij}(\overline{u}, \overline{v})} \right) $
  \end{enumerate}
\end{remark}