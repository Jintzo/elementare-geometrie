\chapter{Geometrie von Flächen}

Ziel dieses Kapitels ist es, auf zweidimensionalen Mannigfaltigkeiten (beziehungsweise Flächen im $ \R^3 $) ``Geometrie'' zu betreiben  (also beispielsweise Längen und Winkel messen und so weiter).\footnote{Für mehr Informationen: \textbf{Gauß} (1827) vgl. \textbf{Spirak}: \emph{A comprehensive introduction to differential geometry} Vol. III --- how to read Gauß}

\section{Reguläre Flächen in $ \R^3 $}

\begin{remark}[Erinnerung]
  Eine Teilmenge $ S \subset \R^3 $ ist eine \term{reguläre Fläche}, falls es zu jedem Punkt $ p \in S $ eine offene Umgebung $ V $ von $ p $ in $ \R^3 $, eine offene Teilmenge $ U \subset \R^2 $ und eine $ C^\infty $-Abbildung
  \begin{equation*}
    x: U \ni (u,v) \mapsto x(u,v) = \left( x_1(u,v),x_2(u,v),x_3(u,v) \right) \in \R^3
  \end{equation*}
  gibt mit:
  \begin{enumerate}
    \item $ x(U) = S \cap V $ und $ x: U \to S \cap V $ ist ein Homöomorphismus. 
  \end{enumerate}
\end{remark}