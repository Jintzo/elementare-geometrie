\chapter{Geometrie von Flächen}

Ziel dieses Kapitels ist es, auf zweidimensionalen Mannigfaltigkeiten (beziehungsweise Flächen im $ \R^3 $) ``Geometrie'' zu betreiben  (also beispielsweise Längen und Winkel messen und so weiter).\footnote{Für mehr Informationen: \textbf{Gauß} (1827) vgl. \textbf{Spirak}: \emph{A comprehensive introduction to differential geometry} Vol. III --- how to read Gauß}

\section{Reguläre Flächen in $ \R^3 $}

\begin{remark}[Erinnerung an reguläre Flächen]
  Eine Teilmenge $ S \subset \R^3 $ ist eine \term{reguläre Fläche}, falls es zu jedem Punkt $ p \in S $ eine offene Umgebung $ V $ von $ p $ in $ \R^3 $, eine offene Teilmenge $ U \subset \R^2 $ und eine $ C^\infty $-Abbildung
  \begin{equation*}
    x: U \ni (u,v) \mapsto x(u,v) = \left( x_1(u,v),x_2(u,v),x_3(u,v) \right) \in \R^3
  \end{equation*}
  gibt mit:
  \begin{enumerate}
    \item $ x(U) = S \cap V $ und $ x: U \to S \cap V $ ist ein Homöomorphismus. 
    \item Das Differenzial $ \text{d}x|_{(u,v)}: \underset{\cong \R^2}{T_{(u,v)}\R^2} \to \underset{\cong \R^3}{T_{x(u,v)}\R^3} $ ist injektiv ($ \forall (u,v) \in U $)
    \begin{itemize}
      \item[$ \Leftrightarrow $] Die Funktionalmatrix
        \begin{equation*}
          \begin{pmatrix}
            \frac{\partial x_1}{\partial u}(u,v) & \frac{\partial x_1}{\partial v}(u,v) \\
            \frac{\partial x_2}{\partial u}(u,v) & \frac{\partial x_2}{\partial v}(u,v) \\
            \frac{\partial x_3}{\partial u}(u,v) & \frac{\partial x_3}{\partial v}(u,v)
          \end{pmatrix} \eqqcolon \begin{pmatrix}
            x_u(u,v) & x_v(u,v)
          \end{pmatrix}
        \end{equation*}
        hat Rang $ 2 $ ($ \forall (u,v) \in U $).
      \item[$ \Leftrightarrow $] $ x_u(u,v) $, $ x_v(u,v) $ sind linear unabhängig ($ \forall (u,v) \in U $).
      \item[$ \Leftrightarrow $] Vektorprodukt $ x_u(u,v) \times x_v(u,v) \neq 0 $ ($ \forall (u,v) \in U $).
    \end{itemize}
  \end{enumerate}
\end{remark}

\begin{remark}[Erinnerung an das Kreuz-/Vektorprodukt]
  \ \\
  $ a \coloneqq (a_1, a_2, a_3) \in \R^3 $ und $ b \coloneqq (b_1, b_2, b_3) \in \R^3 $.
  \begin{equation*}
    a \wedge b \ (\cong a \times b) \ \coloneqq (a_2b_3-a_3b_2, a_3b_1-a_1b_3, a_1b_2-a_2b_1) \in \R^3\text{.}
  \end{equation*}
  \emph{Eigenschaften}:
  \begin{enumerate}
    \item $ a \wedge b \bot a $ \quad $ a \wedge b \bot b $ 
    \item $ \det(a, b, a \wedge b) \geq 0 $
    \item $ a \wedge (-b) = - (a \wedge b) $
    \item $ \Vert a \wedge b \Vert = \Vert a \Vert * \Vert b \Vert * \sin \alpha $ (Winkel zwischen den Vektoren), Fläche des von $ a $ und $ b $ aufgespannten Parallelogramms
  \end{enumerate}
\end{remark}

\begin{definition}[Tangentialraum]
  Der \term{Tangentialraum}\label{def:tangentialraum} in Punkt $ p \in \R^3 $ ist der affine Unterraum
  \begin{equation*}
    T_p\R^3 \coloneqq \{ p \} \times \R^3 = \{ (p,v) : v \in \R^3 \}\text{.}
  \end{equation*}
  Für eine reguläre Fläche $ S $ und $ p = x(u,v) \in S $ ist die \term{Tangentialebene}\label{def:tangentialebene} in $ p \in S $ definiert als
  \begin{equation*}
    T_pS \coloneqq \text{d}x_{(u,v)}\left(T_{(u,v)}\R^2\right) \coloneqq \{ p \} \times [x_u(u,v), x_v(u,v)] \subset T_p\R^3
  \end{equation*}
  $ 2 $-dimensionaler, affiner Unterraum des $ \R^3 $.
\end{definition}

\begin{remark}[Geometrische Interpretation des Tangentialraums]
  \
  % TODO Grafik einfügen
  \begin{equation*}
    x_u(u_0,v_0) = \frac{\partial x}{\partial u}(u_0,v_0) = \frac{\text{d}}{\text{d}t}|_{t = 0}x(u_0+t,v_0) = \text{d}x|_{(u_0,v_0)}(e_1)
  \end{equation*}
  \emph{Allgemein}: \\*
  Sei $ c(t) \coloneqq x(u(t),v(t)) $ eine \term{Flächenkurve}\label{def:flaechenkurve} in $ x(U) $ durch den Punkt $ x(u(0),v(0)) = x(u_0,v_0) $. \\*
  \term{Tangentialvektor}\label{def:tangentialvektor} an $ c $ im Punkt $ x(u_0,v_0): $
  \begin{align*}
    c'(0) &= \frac{\text{d}c}{\text{d}t}|_{t=0} = \frac{\text{d}}{\text{d}t}x(u(t),v(t))|_{t = 0} \\
      &= \frac{\partial x}{\partial u}(u(0),v(0))u'(0) + \frac{\partial x}{\partial v}v'(0) = x_u(u_0,v_0)u'(0)+x_v(u_0,v_0)v'(0)
  \end{align*}
  \emph{Also}: Tangentialebene in $ x(u_0,v_0) = $ Menge aller Tangentialvektoren als Flächenkurven.
\end{remark}

\begin{remark}[Parameterisierungsunabhängigkeit obiger Definitionen]
  \ \\
  Sei $ \overline{x} : \overline{U} \to \overline{x}(\overline{U}) = x(U) $ eine andere Parametrisierung von $ S $ um $ p = x(u_0, v_0) = \overline{x}(\overline{u_0}, \overline{v_0}) $. \\
  \emph{Zu zeigen}: Die lineare Hüllen sind gleich: $ [\overline{x}_{\overline{u}}, \overline{x}_{\overline{v}}] = [x_u, x_v] $. \\
  Es ist $ k \coloneqq \overline{x}^{-1}*x : U \to \overline{U} $ die Koordinatentransformation:
  \begin{equation*}
    x_u = \frac{\partial x}{\partial u}(u,v) = \frac{\partial \overline{x}(\overline{x}^{-1} \circ x)}{\partial u}(u,v) = \frac{\partial \overline{x}}{\partial u}(\overline{u}, \overline{v}) = \frac{\partial \overline{x}}{\partial \overline{u}}(\overline{u},\overline{v}) *\frac{\partial \overline{u}}{\partial u}(u,v)+\frac{\partial \overline{x}}{\partial \overline{v}} * \frac{\partial \overline{v}}{\partial u}
  \end{equation*}
  Entsprechend:
  \begin{align*}
    x_u &= \overline{x}_{\overline{u}}\frac{\partial \overline{u}}{\partial u} + \overline{x}_{\overline{v}}\frac{\partial \overline{v}}{\partial u} \quad \text{d.h. $ x_u $ ist Linearkombination von $ \overline{x}_{\overline{u}} $ und $ \overline{x}_{\overline{v}} $} \\
    x_v &= \overline{x}_{\overline{u}}\frac{\partial \overline{u}}{\partial v} + \overline{x}_{\overline{v}}\frac{\partial \overline{v}}{\partial v} \quad \text{d.h. $ x_v $ ist Linearkombination von $ \overline{x}_{\overline{u}} $ und $ \overline{x}_{\overline{v}} $}
  \end{align*}
  Also $ [x_u, x_v] = [\overline{x}_{\overline{u}}, \overline{x}_{\overline{v}}] $, verschiedene Basen von $ T_pS $ mit Basis-Transformations-Matrix
  \begin{equation*}
    D(u,v) = \begin{pmatrix}
      \frac{\partial \overline{u}}{\partial u} & \frac{\partial \overline{u}}{\partial v} \\
      \frac{\partial \overline{v}}{\partial u} & \frac{\partial \overline{v}}{\partial v}
    \end{pmatrix}\text{.}
  \end{equation*}
  Das ist die Funktionalmatrix der Parametertransformation. Insb. ist $ \det D(u,v) \neq 0 $.
\end{remark}

\begin{example}
  \
  \begin{enumerate}
    \item \emph{affine Ebene}: $ a_0, a, b \in \R^3 $, $ S \coloneqq \{ a_0 + ua + vb : u,v \in \R \} $ ist reguläre Fläche, falls $ a $ und $ b $ linear unabhängig sind. Mit
    \begin{equation*}
      U \coloneqq \R^2, \ V \coloneqq \R^3, \ x: \R^2 \ni (u,v) \mapsto a_0 + ua + vb \in \R^3\text{,}
    \end{equation*}
    \begin{equation*}
      x_u  =\frac{\partial x}{\partial u} = a, \ x_v = \frac{\partial x}{\partial v} = b, \ T_{x(u,v)}S = \{ x(u,v) \} \times [a,b] \cong S\text{.}
    \end{equation*}
  \end{enumerate}
\end{example}