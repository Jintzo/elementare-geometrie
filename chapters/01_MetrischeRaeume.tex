\chapter{Einstieg --- Metrische Räume}

\section{Vorbemerkungen}

Inhalt dieser Vorlesung wird sowohl \emph{Stetigkeitsgeometrie} (Topologie) als auch \emph{metrische Geometrie} sein. Die unten abgebildeten Objekte sind im Sinne der Stetigkeitsgeometrie "topologisch äquivalent", im Sinne der metrischen Geometrie sind diese allerdings verschieden.

% TODO Abbildung 1 einfügen

\begin{bla}{Kartographieproblem}
  \begin{marginfigure}
    % TODO Abbildung 2 einfügen
  \end{marginfigure}
  Ein zentrales Problem der Kartographie ist die \emph{längentreue} Abbildung einer Fläche auf der Weltkugel auf eine Fläche auf Papier. Mithilfe der Differentialgeometrie und der Gauß-Krümmung lässt sich zeigen, dass das nicht möglich ist.
\end{bla}

\section{Definitionen zu metrischen Räumen}

\begin{definition}{Metrik}
  Sei $ X $ eine Menge. Eine Funktion $ d: X \times X \to \R_{>0} $ ist eine \emph{Metrik} (Abstandsfunktion), falls $ \forall x, y, z \in X $ gilt:
  \begin{enumerate}
    \item \textbf{Positivität}: $ d(x, y) = 0 \Leftrightarrow x = y $ 
    \item \textbf{Symmetrie}: $ d(x,y) = d(y,x) $
    \item \textbf{Dreiecksungleichung}: $ d(x,z) \leq d(x,y) + d(y,z) $
  \end{enumerate}
\end{definition}

\begin{definition}{Metrischer Raum}
  Ein \emph{metrischer Raum} ist ein Paar $ (X,d) $ aus einer Menge und einer Metrik auf dieser.
\end{definition}

\begin{definition}{Pseudometrik}
   Eine \emph{Pseudometrik} erfüllt die gleichen Bedingungen wie eine Metrik, außer $ d(x,y) = 0 \Rightarrow x = y $ --- die Umkehrung gilt.
\end{definition}

\begin{definition}{Abgeschlossener $ k $-Ball von $ x $}
  Eine Teilmenge $ \overline{B_r(x)} \coloneqq \{ y \in X : d(x,y) \leq r \} $ heißt \emph{abgeschlossener $ r $-Ball um $ x $}.
\end{definition}

\begin{definition}{Abstandserhaltende Abbildung}
  Sind $ (X, d_X) $ und $ (Y, d_Y) $ metrische Räume, so heißt eine Abbildung $ f: X \to Y $ \emph{abstandserhaltend}, falls
  \begin{equation*}
    \forall x, y \in X: d_Y(f(x), f(y)) = d_X(x, y)\text{.}
  \end{equation*}
\end{definition}

\begin{definition}{Isometrie}
  Eine \emph{Isometrie} ist eine bijektive, abstandserhaltende Abbildung. \\
  Falls eine Isometrie $ f: (X, d_X) \to (Y, d_Y) $ existiert, so heißen $ X $ und $ Y $ \emph{isometrisch}.
\end{definition}

\section{Beispiele zu metrischen Räumen}

\begin{example}{Triviale Metrik}
  Menge $ X $, $ d(x, y) \coloneqq \begin{cases}
    0, &x = y \\
    1, & x \neq y
  \end{cases} \leadsto $ jede Menge lässt sich zu einer Metrik verwursten. 
\end{example}

\begin{example}{Simple Metriken}
  \begin{marginfigure}
    \textbf{Anmerkung}: Wenn $ d(x, y) $ eine Metrik ist, so ist auch $ \widetilde{d}(x, y) \coloneqq \lambda d(x, y) $ mit $ \lambda \in \R_{>0} $ eine Metrik.
  \end{marginfigure}
  Sei $ X = \R $.
  \begin{itemize}
    \item $ d_1(s, t) \coloneqq |s-t| $ ist Metrik.
    \item $ d_2(s, t) \coloneqq \log(|s-t|+1) $ ist Metrik.
  \end{itemize}
\end{example}

\begin{example}{Standardmetrik}
  \label{ex:standardmetrik}
  $ X = \R^n $, $ d_e(x, y) \coloneqq \sqrt{\sum_{i=1}^n(x_i-y_i)^2} = ||x-y|| $ ist die (euklidische) Standardmetrik auf dem $ \R^n $. Die Dreiecksungleichung folgt aus der Cauchy-Schwarz-Ungleichung\sidenote{\textbf{Cauchy-Schwarz-Ungleichung}: \\ $ \langle x, y \rangle \leq ||x||*||y|| \quad (x, y \in \R) $}. \\ \ \\
  \textbf{Bemerkung} (aus LA II): Isometrien von $ (\R^n, d_e) $ sind Translationen, Rotationen und Spiegelungen.
\end{example}

\begin{example}{Maximumsmetrik}
  \label{ex:maximumsmetrik}
  $ X = \R $, $ d(x, y) \coloneqq \underset{1 \leq i \leq n}{\max} |x_i-y_i| $ ist Metrik.
\end{example}

\begin{example}{\autoref{ex:standardmetrik} und \autoref{ex:maximumsmetrik} allgemein: Norm}
  $ V $ sei $ \R $-Vektorraum. Eine \emph{Norm} auf $ V $ ist eine Abbildung $ ||\cdot|| : V \to \R_{>0} $, so dass $ \forall v, w \in V, \lambda \in \R $:
  \begin{enumerate}
    \item \textbf{Definitheit}: $ ||v|| = 0 \Leftrightarrow v = 0 $
    \item \textbf{absolute Homogenität}: $ ||\lambda v|| = |\lambda| * ||v|| $
    \item \textbf{Dreiecksungleichung}: $ ||v+w|| \leq ||v||+||w|| $
  \end{enumerate}
  Eine Norm definiert eine Metrik durch $ d(v, w) \coloneqq ||v-w|| $.
\end{example}

\begin{example}{Einheitssphären}
  $ S_1^n \coloneqq \{ x \in \R^{n+1} : ||x|| = 1 \} $  ist die $ n $-te \emph{Einheitssphäre}. \\
  Auf dieser ist mit $ d_W(x, y) \coloneqq \arccos(\langle x, y \rangle) $ die \emph{Winkel-Metrik} definiert.
\end{example}

\begin{example}{Hamming-Metrik}
  Es ist $ \mathbb{F}_2 $ der Körper mit zwei Elementen $ \{ 0, 1 \} $,
  \begin{equation*}
    X \coloneqq \mathbb{F}_2^n = \{ (f_1, \dots, f_n) : f_i = 0 \vee f_i = 1 \ (i \in {1, \dots, n}) \}
  \end{equation*}
  die Menge der binären Zahlenfolgen der Länge $ n $. Die \emph{Hamming-Metrik} ist definiert als
  \begin{equation*}
    d_H: X \times X \to \R_{>0}, \quad d_H(u,v) = |\{ i : u_i \neq v_i \}|\text{.}
  \end{equation*}
\end{example}