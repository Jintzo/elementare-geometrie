\chapter{Spezielle Klassen von topologischen Räumen}

\section{Übersicht}

Folgende spezielle Klassen sollen diskutiert werden:
\begin{itemize}
  \item \hyperref[def:metrischerRaum]{metrische Räume} $ \leadsto $ metrische Geometrie
  \item \hyperref[def:topologischeMannigfaltigkeit]{Mannigfaltigkeiten} (Grundobjekte in Differenzialgeometrie, Physik,\dots)
  \item \hyperref[def:polyeder]{Polyeder}, \hyperref[def:simplizialkomplex]{Simplizialkomplexe} (Kombinatorik, algebraische Topologie)
  \item \hyperref[def:bahnenraum]{Bahnen-Räume} von \hyperref[def:gruppenaktion]{Gruppenaktionen} (geometrische Gruppentheorie)
\end{itemize}

\section{Topologische Mannigfaltigkeiten}

\begin{definition}[Topologische Mannigfaltigkeit]
  \label{def:topologischeMannigfaltigkeit}
  Eine \term{topologische Mannigfaltigkeit} ist ein \hyperref[def:topologie]{topologischer Raum} $ M $ mit folgenden Eigenschaften:
  \begin{enumerate}
    \item $ M $ ist \term{lokal euklidisch}\label{def:lokalEuklidisch}, d.h. $ \forall p \in M \ \exists $ offene Umgebung $ U $ von $ p $ und ein \hyperref[def:homoeomorphismus]{Homöomorphismus} $ \varphi: U \to \varphi(U) \subset \R^n $ mit festem $ n $. Das Paar $ (\varphi, U) $ heißt \term{Karte}\label{def:karte}\footnote{Eine mathematische Karte ist einer echten Karte ähnlich. Man nehme einen Punkt, zum Beispiel Karlsruhe, und beschreibt die Umgebung von Karlsruhe in Form einer Karte auf einer DIN A4-Karte. Das ist natürlich nicht bijektiv, aber man versucht es möglichst bijektiv zu machen.} und $ \mathcal{A} = \left\{ (\varphi_a, U_\alpha) : \alpha \in A \right\} $ mit $ \bigcup_{\alpha \in A}U_\alpha = M $ heißt \term{Atlas}\label{def:atlas}.
    \item $ M $ ist \hyperref[def:hausdorffsch]{hausdorffsch} und besitzt abzählbare Basis der Topologie.
  \end{enumerate}
  \emph{Bemerkung}:
  \begin{itemize}
    \item Die zweite Eigenschaft ist ``technisch'' und garantiert , dass eine ``Zerlegung der Eins'' existiert (braucht man z.B. für die Existenz von \hyperref[def:riemannscheMetrik]{Riemannschen Metriken}). 
    \item Die Zahl $ n $ heißt \term{Dimension}\label{def:dimension} von $ M $ (eindeutig, wenn $ M $ \hyperref[def:zusammenhaengend]{zusammenhängend} ist, siehe \hyperref[th:satzGebietstreue]{Satz von Gebietstreue}). 
  \end{itemize}
\end{definition}

\begin{example}[{{\hyperref[def:topologischeMannigfaltigkeit]{Topologische Mannigfaltigkeiten}}}]
  \
  \begin{enumerate}
    \setcounter{enumi}{-1}

    \item Eine abzählbare Menge mit \hyperref[bsp:diskreteTopologie]{diskreter Topologie} (jeder Punkt ist offen) ist eine $ 0 $-dimensionale Mannigfaltigkeit.

    \item $ S^1 $ ist eine kompakte, zusammenhängenge $ 1 $-dimensionale Mannigfaltigkeit. \\
      $ \R $ ist \hyperref[def:kompakt]{nichtkompakte}, \hyperref[def:zusammenhaengend]{zusammenhängende} $ 1 $-Mannigfaltigkeit.

    \item Jede offene Teilmenge einer Mannigfaltigkeit ist wieder eine Mannigfaltigkeit, z.B. ist jede offene Teilmenge von $ \R^n $ eine $ n $-dimensionale Mannigfaltigkeit (hier ist Karte = Einschränkung der Identität). \\
    \emph{Spezialfall}: $ \text{GL}(n,\R) = \{ A \in \R^{n \times n} : \det A \neq 0 \} $ ist offene Teilmenge von $ \R^{n^2} $, also eine $ n^2 $-dimensionale Mannigfaltigkeit, denn:
    \begin{itemize}
      \item $ \det : \R^{n \times n} \to \R $ ist stetig
      \item $ \{ 0 \} $ ist abgeschlossen in $ \R $ 
      \item $ \det^{-1} \{ 0 \} $ ist abgeschlossen in $ \R^{n \times n} $
      \item $ \R^{n \times n} \setminus \det^{-1} \{ 0 \} = \text{GL}(n,\R ) $ ist offen in $ \R^{n \times n} $
    \end{itemize}

    \item Die $ n $-dimensionale Sphäre mit Radius $ R > 0 $,
    \begin{equation*}
      S^n_R = \{ x \in \R^{n-1} : \Vert x \Vert = R \}\text{,}
    \end{equation*}
    ist $ n $-dimensionale topologische Mannigfaltigkeit.
    \begin{proof}
      Sei $ (x_1, \dots, x_{n+1}) = p \in S^n_R $, oBdA $ x_{n+1} > 0 $. Man betrachte die Abbildung
      \begin{align*}
        \phi^{-1} : D^n_R \coloneqq \left\{ x \in \R^n : \Vert x \Vert < R \right\} &\to \phi(D^n_R) \subset S^n_R \\
          (x_1, \dots, x_n) &\mapsto \left(x_1, \dots, x_n, \sqrt{R^2-(x_1^2 + \cdots + x_n^2)}\right)
      \end{align*}
      d.h. $ \phi $ ist Einschränkung der Orthogonalprojektion
      \begin{align*}
        \R^{n+1} &\to \R^n \subset \R^{n+1} \\
          (x_1, \dots, x_{n+1}) &\mapsto (x_1, \dots, x_n, 0)
      \end{align*}
      auf $ S_R^n $. \\
      Alternativ kann via stereographischer Projektion mit $ 2 $ Karten ausgekommen werden. \\
      Ein Atlas mit einer Karte existiert nicht.
    \end{proof}
    \item Das Produkt von $ n_1 $-dimensionaler Mannigfaltigkeit $ M_1 $ und $ n_2 $-dimensionaler Mannigfaltigkeit $ M_2 $ ist $ (n_1+n_2) $-dimensionale Mannigfaltigkeit. \\
    \emph{Karten}: $ (p_1, p_2) \in M_1 \times M_2 $,
    \begin{equation*}
      \tilde{\phi} : U_1 \times U_2 \to \phi_1(U_1) \times \phi_2(U_2) \subset \R^{n_1} \times \R^{n_2}
    \end{equation*}
    mit $ (U_1, \phi_1) $ Karte von $ M_1 $ um $ p_1 $ und $ (U_2, \phi_2) $ Karte von $ M_2 $ um $ p_2 $.
  \end{enumerate}
\end{example}

\begin{remark}[``Wieviele topologische Mannigfaltigkeit gibt es?'']
  \
  \begin{itemize}
    \item \emph{Dimension} $ n=1 $: Im wesentlichen $ \R $ (nicht kompakt) oder $ S^1 $ (kompakt).
    \item \emph{Dimension} $ n=2 $: Liste für zusammenhängende, kompakte, ``orientierbare'', ``randlose'' Mannigfaltigkeiten:
    \begin{itemize}
      \item $ g = 0 $: $ S^2 $ Einheitssphäre
      \item $ g = 1 $: $ T^2 = S^1 \times S^1 $ Torus
      \item $ g = 2 $: Brezel
      \item \dots 
    \end{itemize}
    $ g $ ist das \term{Geschlecht}\label{def:geschlecht} der Mannigfaltigkeit.
    \item \emph{Dimension} $ n=3 $: Thurston's \term{Geometrisierungs-Vermutung}\label{theorem:geometrisierungsvermutung} ($ \sim $ 1978) \\
      Bewiesen von Perelman (2002), ein Milleniumsproblem.
    \item \emph{Dimension} $ n \geq 4 $: Allgemeine Klassifikation unmöglich, weil das Homöomorphieproblem hier nicht entscheidbar ist (Markov, 1960).
  \end{itemize}
\end{remark}

\section{Differenzierbare Mannigfaltigkeiten}

\begin{definition}[Kartenwechsel, differenzierbare Mannigfaltigkeit]
  Sei $ M $ topologische Mannigfaltigkeit, $ p \in M $. Ein \term{Kartenwechsel}\label{def:kartenwechsel} ist ein Homöomorphismus
  \begin{equation*}
    \psi \circ \phi^{-1}: \underbrace{\phi(D)}_{\subset \R^n} \to \underbrace{\psi(D)}_{\subset \R^n}\text{.}
  \end{equation*}
  Ein Atlas $ \mathcal{A} $ von $ M $ ist ein \term{$ C^\infty $-Atlas}\label{def:dbatlas}, falls alle möglichen Kartenwechsel $ C^\infty $-Abbildungen von $ \R^n $ sind, also alle partiellen Ableitungen existieren und stetig sind. \\
  Ein maximaler $ C^\infty $-Atlas heißt \term{$ C^\infty $-Struktur}\label{def:dbstruktur} auf der topologischen Mannigfaltigkeit $ M $. Eine $ C^\infty $-Mannigfaltigkeit ist eine topologische Mannigfaltigkeit mit einer $ C^\infty $-Struktur (auch \term{glatte}\label{def:glattemannigfaltigkeit} oder \term{differenzierbare Mannigfaltigkeit}\label{def:dbmannigfaltigkeit}).
\end{definition}

\begin{remark}
  \
  \begin{enumerate}
    \item Es gibt topologische Mannigfaltigkeiten ohne differenzierbare Struktur\footnote{Kerraire 1960}.
    \item Auf $ \R^n $, $ n \neq 4 $\footnote{Kirby, Friedman 1980}, existiert genau eine differenzierbare Struktur.
    \item Auf $ S^7 $ existieren $ 28 $ differenzierbare Strukturen\footnote{Milnor 1956}.
  \end{enumerate}
\end{remark}

Wozu die Differenzierbarkeitsbedingung für Kartenwechsel? Beispielsweise für die Definition von differenzierbaren Abbildungen zwischen differenzierbaren Mannigfaltigkeiten.

\begin{definition}[Differenzierbarkeit]
  Seien $ M^m $, $ N^n $ differenzierbare Mannigfaltigkeiten und $ F: M^m \to N^n $ stetig. $ F $ heißt \term{differenzierbar in $ p \in M $}\label{def:differenzierbar}, falls für Karten $ (U, \phi) $ um $ p $ und $ (V, \psi) $ um $ F(p) $ gilt:
  \begin{equation*}
    \psi \circ F \circ \phi^{-1}: \underbrace{\phi(U)}_{\subset \R^m} \to \underbrace{\psi(V)}_{\subset \R^n}
  \end{equation*}
  ist $ C^\infty $-Abbildung in $ \phi(p) $. \\
  So kommt man von einem abstrakten $ F $ zwischen den Mannigfaltigkeiten zu einer konkreten Darstellung von $ F $.
\end{definition}
