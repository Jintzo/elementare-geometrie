\chapter{Spezielle Klassen von topologischen Räumen}

\section{Übersicht}

Folgende spezielle Klassen sollen diskutiert werden:
\begin{itemize}
  \item \hyperref[def:metrischerRaum]{metrische Räume} $ \leadsto $ metrische Geometrie
  \item \hyperref[def:topologischeMannigfaltigkeit]{Mannigfaltigkeiten} (Grundobjekte in Differenzialgeometrie, Physik,\dots)
  \item \hyperref[def:polyeder]{Polyeder}, \hyperref[def:simplizialkomplex]{Simplizialkomplexe} (Kombinatorik, algebraische Topologie)
  \item \hyperref[def:bahnenraum]{Bahnen-Räume} von \hyperref[def:gruppenaktion]{Gruppenaktionen} (geometrische Gruppentheorie)
\end{itemize}

\section{Topologische Mannigfaltigkeiten}

\begin{definition}[Topologische Mannigfaltigkeit]
  \label{def:topologischeMannigfaltigkeit}
  Eine \term{topologische Mannigfaltigkeit} ist ein \hyperref[def:topologie]{topologischer Raum} $ M $ mit folgenden Eigenschaften:
  \begin{enumerate}
    \item $ M $ ist \term{lokal euklidisch}\label{def:lokalEuklidisch}, d.h. $ \forall p \in M \ \exists $ offene Umgebung $ U $ von $ p $ und ein \hyperref[def:homoeomorphismus]{Homöomorphismus} $ \varphi: U \to \varphi(U) \subset \R^n $ mit festem $ n $. Das Paar $ (\varphi, U) $ heißt \term{Karte}\label{def:karte}\footnote{Eine mathematische Karte ist einer echten Karte ähnlich. Man nehme einen Punkt, zum Beispiel Karlsruhe, und beschreibt die Umgebung von Karlsruhe in Form einer Karte auf einer DIN A4-Karte. Das ist natürlich nicht bijektiv, aber man versucht es möglichst bijektiv zu machen.} und $ \mathcal{A} = \left\{ (\varphi_a, U_\alpha) : \alpha \in A \right\} $ mit $ \bigcup_{\alpha \in A}U_\alpha = M $ heißt \term{Atlas}\label{def:atlas}.
    \item $ M $ ist \hyperref[def:hausdorffsch]{hausdorffsch} und besitzt abzählbare Basis der Topologie.
  \end{enumerate}
  \emph{Bemerkung}:
  \begin{itemize}
    \item Die zweite Eigenschaft ist ``technisch'' und garantiert , dass eine ``Zerlegung der Eins'' existiert (braucht man z.B. für die Existenz von \hyperref[def:riemannscheMetrik]{Riemannschen Metriken}). 
    \item Die Zahl $ n $ heißt \term{Dimension}\label{def:dimension} von $ M $ (eindeutig, wenn $ M $ \hyperref[def:zusammenhaengend]{zusammenhängend} ist, siehe \hyperref[th:satzGebietstreue]{Satz von Gebietstreue}). 
  \end{itemize}
\end{definition}

\begin{example}[{{\hyperref[def:topologischeMannigfaltigkeit]{Topologische Mannigfaltigkeiten}}}]
  \
  \begin{enumerate}
    \setcounter{enumi}{-1}

    \item Eine abzählbare Menge mit \hyperref[bsp:diskreteTopologie]{diskreter Topologie} (jeder Punkt ist offen) ist eine $ 0 $-dimensionale Mannigfaltigkeit.

    \item $ S^1 $ ist eine \hyperref[def:kompakt]{kompakte}, \hyperref[def:zusammenhaengend]{zusammenhängenge} $ 1 $-dimensionale Mannigfaltigkeit. \\
      $ \R $ ist nichtkompakte, zusammenhängende $ 1 $-Mannigfaltigkeit.

    \item Jede offene Teilmenge einer Mannigfaltigkeit ist wieder eine Mannigfaltigkeit, z.B. ist jede offene Teilmenge von $ \R^n $ eine $ n $-dimensionale Mannigfaltigkeit (hier ist \hyperref[def:karte]{Karte} = Einschränkung der Identität). \\
    \emph{Spezialfall}: $ \text{GL}(n,\R) = \{ A \in \R^{n \times n} : \det A \neq 0 \} $ ist offene Teilmenge von $ \R^{n^2} $, also eine $ n^2 $-dimensionale Mannigfaltigkeit, denn:
    \begin{itemize}
      \item $ \det : \R^{n \times n} \to \R $ ist stetig
      \item $ \{ 0 \} $ ist abgeschlossen in $ \R $ 
      \item $ \det^{-1} \{ 0 \} $ ist abgeschlossen in $ \R^{n \times n} $
      \item $ \R^{n \times n} \setminus \det^{-1} \{ 0 \} = \text{GL}(n,\R ) $ ist offen in $ \R^{n \times n} $
    \end{itemize}

    \item Die \hyperref[bsp:einheitssphaere]{$ n $-dimensionale Sphäre} mit Radius $ R > 0 $,
    \begin{equation*}
      S^n_R = \{ x \in \R^{n-1} : \Vert x \Vert = R \}\text{,}
    \end{equation*}
    ist $ n $-dimensionale topologische Mannigfaltigkeit.
    \begin{proof}
      Sei $ (x_1, \dots, x_{n+1}) = p \in S^n_R $, oBdA $ x_{n+1} > 0 $. Man betrachte die Abbildung
      \begin{align*}
        \phi^{-1} : D^n_R \coloneqq \left\{ x \in \R^n : \Vert x \Vert < R \right\} &\to \phi(D^n_R) \subset S^n_R \\
          (x_1, \dots, x_n) &\mapsto \left(x_1, \dots, x_n, \sqrt{R^2-(x_1^2 + \cdots + x_n^2)}\right)
      \end{align*}
      d.h. $ \phi $ ist Einschränkung der Orthogonalprojektion
      \begin{align*}
        \R^{n+1} &\to \R^n \subset \R^{n+1} \\
          (x_1, \dots, x_{n+1}) &\mapsto (x_1, \dots, x_n, 0)
      \end{align*}
      auf $ S_R^n $. \\
      Alternativ kann via stereographischer Projektion mit $ 2 $ Karten ausgekommen werden. \\
      Ein \hyperref[def:atlas]{Atlas} mit einer Karte existiert nicht.
    \end{proof}
    \item Das Produkt von $ n_1 $-dimensionaler Mannigfaltigkeit $ M_1 $ und $ n_2 $-dimensionaler Mannigfaltigkeit $ M_2 $ ist $ (n_1+n_2) $-dimensionale Mannigfaltigkeit. \\
    \emph{Karten}: $ (p_1, p_2) \in M_1 \times M_2 $,
    \begin{equation*}
      \widetilde{\phi} : U_1 \times U_2 \to \phi_1(U_1) \times \phi_2(U_2) \subset \R^{n_1} \times \R^{n_2}
    \end{equation*}
    mit $ (U_1, \phi_1) $ Karte von $ M_1 $ um $ p_1 $ und $ (U_2, \phi_2) $ Karte von $ M_2 $ um $ p_2 $.
  \end{enumerate}
\end{example}

\begin{remark}[``Wieviele {\hyperref[def:topologischeMannigfaltigkeit]{topologische Mannigfaltigkeiten}} gibt es?'']
  \
  \begin{itemize}
    \item \emph{Dimension} $ n=1 $: Im wesentlichen $ \R $ (nicht \hyperref[def:kompakt]{kompakt}) oder $ S^1 $ (kompakt).
    \item \emph{Dimension} $ n=2 $: Liste für \hyperref[def:zusammenhaengend]{zusammenhängende}, kompakte, ``orientierbare'', ``randlose'' Mannigfaltigkeiten:
    \begin{itemize}
      \item $ g = 0 $: $ S^2 $ \hyperref[bsp:einheitssphaere]{Einheitssphäre}
      \item $ g = 1 $: $ T^2 = S^1 \times S^1 $ Torus
      \item $ g = 2 $: Brezel
      \item \dots 
    \end{itemize}
    $ g $ ist das \term{Geschlecht}\label{def:geschlecht} der Mannigfaltigkeit.
    \item \emph{Dimension} $ n=3 $: Thurston's \term{Geometrisierungs-Vermutung}\label{theorem:geometrisierungsvermutung} ($ \sim $ 1978) \\
      Bewiesen von Perelman (2002), ein Milleniumsproblem.
    \item \emph{Dimension} $ n \geq 4 $: Allgemeine Klassifikation unmöglich, weil das Homöomorphieproblem hier nicht entscheidbar ist (Markov, 1960).
  \end{itemize}
\end{remark}

\section{Differenzierbare Mannigfaltigkeiten}

\begin{definition}[Kartenwechsel, differenzierbare Mannigfaltigkeit]
  Sei $ M $ \hyperref[def:topologischeMannigfaltigkeit]{topologische Mannigfaltigkeit}, $ p \in M $. Ein \term{Kartenwechsel}\label{def:kartenwechsel} ist ein \hyperref[def:homoeomorphismus]{Homöomorphismus}
  \begin{equation*}
    \psi \circ \phi^{-1}: \underbrace{\phi(D)}_{\subset \R^n} \to \underbrace{\psi(D)}_{\subset \R^n}\text{.}
  \end{equation*}
  Ein \hyperref[def:atlas]{Atlas} $ \mathcal{A} $ von $ M $ ist ein \term{$ C^\infty $-Atlas}\label{def:dbatlas}, falls alle möglichen Kartenwechsel $ C^\infty $-Abbildungen von $ \R^n $ sind, also alle partiellen Ableitungen existieren und stetig sind. \\
  Ein maximaler $ C^\infty $-Atlas heißt \term{$ C^\infty $-Struktur}\label{def:dbstruktur} auf der topologischen Mannigfaltigkeit $ M $. Eine $ C^\infty $-Mannigfaltigkeit ist eine topologische Mannigfaltigkeit mit einer $ C^\infty $-Struktur (auch \term{glatte}\label{def:glattemannigfaltigkeit} oder \term{differenzierbare Mannigfaltigkeit}\label{def:dbmannigfaltigkeit}).
\end{definition}

\begin{remark}
  \
  \begin{enumerate}
    \item Es gibt \hyperref[def:topologischeMannigfaltigkeit]{topologische Mannigfaltigkeiten} ohne \hyperref[def:dbstruktur]{differenzierbare Struktur}\footnote{Kerraire 1960}.
    \item Auf $ \R^n $, $ n \neq 4 $\footnote{Kirby, Friedman 1980}, existiert genau eine differenzierbare Struktur.
    \item Auf $ S^7 $ existieren $ 28 $ differenzierbare Strukturen\footnote{Milnor 1956}.
  \end{enumerate}
  \emph{Frage}: Wozu die Differenzierbarkeitsbedingung für \hyperref[def:kartenwechsel]{Kartenwechsel}? Beispielsweise für die Definition von differenzierbaren Abbildungen zwischen \hyperref[def:dbmannigfaltigkeit]{differenzierbaren Mannigfaltigkeiten}.
\end{remark}


\begin{definition}[Differenzierbarkeit]
  Seien $ M^m $, $ N^n $ \hyperref[def:dbmannigfaltigkeit]{differenzierbare Mannigfaltigkeiten} und $ F: M^m \to N^n $ stetig. $ F $ heißt \term{differenzierbar in $ p \in M $}\label{def:differenzierbar}, falls für \hyperref[def:karte]{Karten} $ (U, \phi) $ um $ p $ und $ (V, \psi) $ um $ F(p) $ gilt:
  \begin{equation*}
    \psi \circ F \circ \phi^{-1}: \underbrace{\phi(U)}_{\subset \R^m} \to \underbrace{\psi(V)}_{\subset \R^n}
  \end{equation*}
  ist $ C^\infty $-Abbildung in $ \phi(p) $. \\
  So kommt man von einem abstrakten $ F $ zwischen den Mannigfaltigkeiten zu einer konkreten Darstellung von $ F $. \\
  $ F $ heißt \term{differenzierbar} ($ C^\infty $), falls $ F $ differenzierbar ist für alle $ p \in M $.
  % TODO Abbildung 1
\end{definition}

\begin{remark}[Wohldefiniertheit der Differenzierbarkeit]
  Differenzierbarkeit in $ p $ ist wohldefiniert (also unabhängig von der Wahl der Karten)
  \begin{proof}
    \emph{Erster Test}: $ \psi \circ F \circ \phi^{-1} $, \emph{zweiter Test} $ \widetilde{\psi} \circ F \circ \widetilde{\phi}^{-1} $ \\
    Es gilt:
    \begin{align*}
      \psi \circ F \circ \phi^{-1} &= \psi \circ \underbrace{\widetilde{\psi}^{-1} \circ \widetilde{\psi}}_{\text{Id}_{\R^n}} \circ F \circ \underbrace{\widetilde{\phi}^{-1} \circ \widetilde{\phi}}_{\text{Id}_{\R^n}} \circ \phi^{-1} \\
        &= \underbrace{\left( \psi \circ \widetilde{\psi}^{-1} \right)}_{C^\infty} \circ \left( \widetilde{\psi} \circ F \circ \widetilde{\phi}^{-1} \right) \circ \underbrace{\left( \widetilde{\phi} \circ \phi^{-1} \right)}_{\text{Kartenwechsel}}
    \end{align*}
    Also: Abbildung in Test 1 ist $ C^\infty $ $ \Leftrightarrow $ Abbildung in Test 2 ist $ C^\infty $.
  \end{proof}
\end{remark}

\clearpage

\begin{remark}
  \
  \begin{itemize}
    \item $ N = \R $, $ F:M \to \R $ (differenzierbar) heißt \term{differenzierbare Funktion}.
    \item $ M = \R $ (oder $ I \subset \R $), $ F: I \to N $ heißt \term{differenzierbare Kuve}.
    \item Eine Abbildung $ F: M \to N $ zwischen differenzierbaren Mannigfaltigkeiten heißt \term{Diffeomorphismus}\label{def:diffeomorphismus}, falls $ F $ bijektiv und $ F $ und $ F^{-1} $ differenzierbar sind (also $ C^\infty $).
    \item Ein Homöomorphismus ist nicht unbedingt ein Diffeomorphismus. Beispielsweise $ \R $ mit Id als Karte, $ F : \R \to \R $, $ x \mapsto x^3 $ ist Homöomorphismus, aber kein Diffeomorphismus, da $ F^{-1} : x \mapsto \sqrt[3]{x} $ ist nicht $ C^\infty $.
    \item Die Menge der Diffeomorphismen einer differenzierbaren Mannigfaltigkeit ist eine Gruppe mit der Verkettung von Abbildungen.
  \end{itemize}
\end{remark}

\begin{example}
  \
  \begin{enumerate}
    \item $ U \subseteq \R^n $ offen (bzgl. Standard-Topologie). \\
      $ \phi_0 \coloneqq \text{Id}\mid_U $ mit zugehörigem maximalen Atlas definiert $ C^\infty $-Struktur auf $ U $, die kanonische differenzierbare Struktur.
    \item $ 2 $-dimensionale Mannigfaltigkeiten heißen auch \term{Flächen}\label{def:flaeche}, speziell \emph{regulär parametrisierte Flächen}\footnote{Gegenstand der klassischen Differentialgeometrie, siehe auch Kapitel 5}.
  \end{enumerate}
\end{example}

\begin{definition}[Reguläre Fläche]
  Eine Teilmenge $ S $ von $ \R^3 $ (mit Teilraum-Topologie von $ \R^3 $) heißt \term{reguläre Fläche}\label{def:regulaereFlaeche}, falls für jeden Punkt $ p \in S $ eine Umgebung $ V $ von $ p $ in $ \R^3 $ und eine Abbildung
  \begin{align*}
    F : \underset{\text{offen}}{U} \subset \R^2 &\to \underset{\text{offene TM von S}}{V \cap S} \subset \R^3 \\
      (u,v) &\mapsto (x(u,v),y(u,v),z(u,v))
  \end{align*}
  existiert, so dass gilt:
  % TODO Abbildung 2
  \begin{enumerate}
    \item $ F $ ist ein differenzierbar Homöomorphismus
    \item das Differential (Jacobi-Matrix) von $ F $,
    \begin{equation*}
       \text{d}F_q : \R^2 \supseteq T_qU \to T_{F(q)}\R^3 \cong \R^3
     \end{equation*} 
     ist \emph{injektiv} (d.h. Jacobi-Matrix hat Rang $ 2 $) für $ \forall q \in U $.
  \end{enumerate}
  $ F $ heißt \term{lokale Parametrisierung}\label{def:lokaleParametrisierung} von $ S $.
\end{definition}

\begin{example}[Rotationsfläche]
  Gegeben ist eine ebene Kurve $ c(v) = \left( r(v), 0, h(v) \right) $, $ v \in [a,b] $ mit $ r(v) > 0 $, $ c'(v) = \left( r'(v), 0, h'(v) \right) $ Tangentialvektor (mit $ C^\infty $-Funktionen $ r $, $ h $).\footnote{$ \Vert c'(v) \Vert \neq 0 \Leftrightarrow (r')^2 + (h')^2 \neq 0 $}
  % TODO Abbildung 3
  \begin{equation*}
    F(u,v) \coloneqq \begin{pmatrix}
      r(v)\cos u \\
      r(v)\sin u \\
      h(v)
    \end{pmatrix}
  \end{equation*}
  ist reguläre Fläche.\footnote{Übung!} \\
  \emph{Beispiel}: $ 2 $-Sphäre von Radius $ R $:
  \begin{equation*}
    (u,v) \mapsto \begin{pmatrix}
      R\cos v \cos u \\
      R\cos v\sin u \\
      R\sin v
    \end{pmatrix}\text{.}
  \end{equation*}
  Es gibt andere Parametrisierungen, beispielsweise
  \begin{equation*}
    (u,v) \mapsto \begin{pmatrix}
      u \\ v \\ \sqrt{R^2-u^2-v^2}
    \end{pmatrix}
  \end{equation*}
  % TODO Abildung 4
\end{example}

\begin{remark}[Geometrische Eigenschaften parametrisierungsunabhängig]
  \ \\
  Geometrische Eigenschaften sollten unabhängig sein von Parametrisierung. Das wird durch Eigenschaft 2 von regulären Flächen garantiert. Genauer gilt: Parameterwechsel sind differenzierbar ($ \leadsto $ reguläre Flächen sind differenzierbare $ 2 $-dimensionale Mannigfaltigkeiten mit $ F^{-1} $ (Umkehr-Abbildung der Parametrisierung) als Karten): \\
  Sei $ p \in S $ und $ F_1 : \R^2 \supseteq U \to S $, $ F_2: \R^2 \supseteq V \to S $ zwei Parametrisierungen, sodass $ p \in F_1(U) \cap F_2(V) \eqqcolon W $. \\
  % TODO Abbildung 5
  \emph{Behauptung}: Der Parameterwechsel 
  \begin{equation*}
    H \coloneqq F_1^{-1} \circ F_2 : \R^2 \supset F_2^{-1}(W) \to F_1^{-1}(W) \subset \R^2
  \end{equation*}
  ist Diffeomorphismus.
  \begin{proof}
    $ H $ ist Homöomorphismus, da $ F_1 $ und $ F_2 $ Homöomorphismen sind. \\
    \emph{Problem}: $ F_1^{-1} $ ist auf einer offenen Teilmenge von $ S $ definiert und und da weiß man nicht was \emph{differenzierbar} heißt. \\
    \emph{Ausweg}: Erweiterung von $ F $. Sei $ r \in F_2^{-1}(W) $ und $ q \coloneqq H(r) $. Da
      \begin{equation*}
        F_1(u,v) = \left( x(u,v),y(u,v),z(u,v) \right)
      \end{equation*}
    reguläre Parametrisierung ist können wir oBdA (erst Koordinatenachsen von $ \R^3 $ umbenennen) annehmen, dass
    \begin{equation*}
      \frac{J(x,y)}{J(u,v)}(q) \neq 0 \quad \text{(Jacobi-Determinante).}
    \end{equation*}
    \emph{Trick}: Erweitere $ F_1 $ zu Abbildung
    \begin{align*}
      \widetilde{F_1} : U \times \R &\to \R^3 \\
        \widetilde{F_1}(u,v,t) &\coloneqq \left( x(u,v),y(u,v),z(u,v)+\bm{t} \right)\text{.}
    \end{align*}
    $ \widetilde{F_1} $ ist differenzierbar und $ \widetilde{F_1}|_{U \times \{ 0 \}} = F_1 $. \\
    Die Jacobi-Determinante von $ \widetilde{F_1} $ in $ (q,0) $,
    \begin{equation*}
      \det \begin{pmatrix}
        \frac{\text{d}x}{\text{d}u} & \frac{\text{d}x}{\text{d}v} & 0 \\
        \frac{\text{d}y}{\text{d}u} & \frac{\text{d}y}{\text{d}v} & 0 \\
        \frac{\text{d}z}{\text{d}u} & \frac{\text{d}z}{\text{d}v} & 1
      \end{pmatrix}(q,0) = \det \begin{pmatrix}
        \frac{\text{d}x}{\text{d}u} & \frac{\text{d}x}{\text{d}v} \\
        \frac{\text{d}y}{\text{d}u} & \frac{\text{d}y}{\text{d}v}
      \end{pmatrix}(q) \neq 0\text{.}
    \end{equation*}
    Nach dem Umkehrsatz (Analysis II) existiert eine Umgebung $ A $ von $ \widetilde{F_1}(q,0) = F_1(q) $ in $ \R^3 $ sodass $ \widetilde{F_1}^{-1} $ auf $ A $ existiert und differenzierbar ($ C^\infty $) ist. Da $ F_2 $ stetig ist existiert Umgebung $ B $ von $ v $ in $ V $, sodass $ F_2(B) \subset A $. Und nun ist $ H|_B = \widetilde{F_1}^{-1} \circ F_2|_B $ ist Verkettung von differenzierbaren Abbildungen, also differenzierbar in $ r $ und da $ r $ beliebig ist ist $ H $ differenzierbar auf $ F_2^{-1}(W) $. 
  \end{proof}
\end{remark}