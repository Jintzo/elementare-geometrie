\chapter{Spezielle Klassen von topologischen Räumen}

\section{Übersicht}

Folgende spezielle Klassen sollen diskutiert werden:
\begin{itemize}
  \item \hyperref[def:metrischerRaum]{metrische Räume} $ \leadsto $ metrische Geometrie
  \item \hyperref[def:topologischeMannigfaltigkeit]{Mannigfaltigkeiten} (Grundobjekte in Differenzialgeometrie, Physik,\dots)
  \item \hyperref[def:polyeder]{Polyeder}, \hyperref[def:simplizialkomplex]{Simplizialkomplexe} (Kombinatorik, algebraische Topologie)
  \item \hyperref[def:bahnenraum]{Bahnen-Räume} von \hyperref[def:gruppenaktion]{Gruppenaktionen} (geometrische Gruppentheorie)
\end{itemize}

\section{Topologische Mannigfaltigkeiten}

\begin{definition}[Topologische Mannigfaltigkeit]
  \label{def:topologischeMannigfaltigkeit}
  Eine \term{topologische Mannigfaltigkeit} ist ein \hyperref[def:topologie]{topologischer Raum} $ M $ mit folgenden Eigenschaften:
  \begin{enumerate}
    \item $ M $ ist \term{lokal euklidisch}\label{def:lokalEuklidisch}, d.h. $ \forall p \in M \ \exists $ offene Umgebung $ U $ von $ p $ und ein \hyperref[def:homoeomorphismus]{Homöomorphismus} $ \varphi: U \to \varphi(U) \subset \R^n $ mit festem $ n $. Das Paar $ (\varphi, U) $ heißt \term{Karte}\label{def:karte}\footnote{Eine mathematische Karte ist einer echten Karte ähnlich. Man nehme einen Punkt, zum Beispiel Karlsruhe, und beschreibt die Umgebung von Karlsruhe in Form einer Karte auf einer DIN A4-Karte. Das ist natürlich nicht bijektiv, aber man versucht es möglichst bijektiv zu machen.} und $ \mathcal{A} = \left\{ (\varphi_a, U_\alpha) : \alpha \in A \right\} $ mit $ \bigcup_{\alpha \in A}U_\alpha = M $ heißt \term{Atlas}\label{def:atlas}.
    \item $ M $ ist \hyperref[def:hausdorffsch]{hausdorffsch} und besitzt abzählbare Basis der Topologie.
  \end{enumerate}
  \emph{Bemerkung}:
  \begin{itemize}
    \item Die zweite Eigenschaft ist ``technisch'' und garantiert , dass eine ``Zerlegung der Eins'' existiert (braucht man z.B. für die Existenz von \hyperref[def:riemannscheMetrik]{Riemannschen Metriken}). 
    \item Die Zahl $ n $ heißt \term{Dimension}\label{def:dimension} von $ M $ (eindeutig, wenn $ M $ \hyperref[def:zusammenhaengend]{zusammenhängend} ist, siehe \hyperref[th:satzGebietstreue]{Satz von Gebietstreue}). 
  \end{itemize}
\end{definition}

\begin{example}[{{\hyperref[def:topologischeMannigfaltigkeit]{Topologische Mannigfaltigkeiten}}}]
  \
  \begin{enumerate}
    \setcounter{enumi}{-1}

    \item Eine abzählbare Menge mit \hyperref[bsp:diskreteTopologie]{diskreter Topologie} (jeder Punkt ist offen) ist eine $ 0 $-dimensionale Mannigfaltigkeit.

    \item $ S^1 $ ist eine \hyperref[def:kompakt]{kompakte}, \hyperref[def:zusammenhaengend]{zusammenhängenge} $ 1 $-dimensionale Mannigfaltigkeit. \\
      $ \R $ ist nichtkompakte, zusammenhängende $ 1 $-Mannigfaltigkeit.

    \item Jede offene Teilmenge einer Mannigfaltigkeit ist wieder eine Mannigfaltigkeit, z.B. ist jede offene Teilmenge von $ \R^n $ eine $ n $-dimensionale Mannigfaltigkeit (hier ist \hyperref[def:karte]{Karte} = Einschränkung der Identität). \\
    \emph{Spezialfall}: $ \text{GL}(n,\R) = \{ A \in \R^{n \times n} : \det A \neq 0 \} $ ist offene Teilmenge von $ \R^{n^2} $, also eine $ n^2 $-dimensionale Mannigfaltigkeit, denn:
    \begin{itemize}
      \item $ \det : \R^{n \times n} \to \R $ ist stetig
      \item $ \{ 0 \} $ ist abgeschlossen in $ \R $ 
      \item $ \det^{-1} \{ 0 \} $ ist abgeschlossen in $ \R^{n \times n} $
      \item $ \R^{n \times n} \setminus \det^{-1} \{ 0 \} = \text{GL}(n,\R ) $ ist offen in $ \R^{n \times n} $
    \end{itemize}

    \item Die \hyperref[bsp:einheitssphaere]{$ n $-dimensionale Sphäre} mit Radius $ R > 0 $,
    \begin{equation*}
      S^n_R = \{ x \in \R^{n+1} : \Vert x \Vert = R \}\text{,}
    \end{equation*}
    ist $ n $-dimensionale topologische Mannigfaltigkeit.
    \begin{proof}
      Sei $ (x_1, \dots, x_{n+1}) = p \in S^n_R $, oBdA $ x_{n+1} > 0 $. Man betrachte die Abbildung
      \begin{align*}
        \phi^{-1} : D^n_R \coloneqq \left\{ x \in \R^n : \Vert x \Vert < R \right\} &\to \phi(D^n_R) \subset S^n_R \\
          (x_1, \dots, x_n) &\mapsto \left(x_1, \dots, x_n, \sqrt{R^2-(x_1^2 + \cdots + x_n^2)}\right)
      \end{align*}
      d.h. $ \phi $ ist Einschränkung der Orthogonalprojektion
      \begin{align*}
        \R^{n+1} &\to \R^n \subset \R^{n+1} \\
          (x_1, \dots, x_{n+1}) &\mapsto (x_1, \dots, x_n, 0)
      \end{align*}
      auf $ S_R^n $. \\
      Alternativ kann via stereographischer Projektion mit $ 2 $ Karten ausgekommen werden. \\
      Ein \hyperref[def:atlas]{Atlas} mit einer Karte existiert nicht.
    \end{proof}
    \item Das Produkt von $ n_1 $-dimensionaler Mannigfaltigkeit $ M_1 $ und $ n_2 $-dimensionaler Mannigfaltigkeit $ M_2 $ ist $ (n_1+n_2) $-dimensionale Mannigfaltigkeit. \\
    \emph{Karten}: $ (p_1, p_2) \in M_1 \times M_2 $,
    \begin{equation*}
      \widetilde{\phi} : U_1 \times U_2 \to \phi_1(U_1) \times \phi_2(U_2) \subset \R^{n_1} \times \R^{n_2}
    \end{equation*}
    mit $ (U_1, \phi_1) $ Karte von $ M_1 $ um $ p_1 $ und $ (U_2, \phi_2) $ Karte von $ M_2 $ um $ p_2 $.
  \end{enumerate}
\end{example}

\begin{remark}[``Wieviele {\hyperref[def:topologischeMannigfaltigkeit]{topologische Mannigfaltigkeiten}} gibt es?'']
  \
  \begin{itemize}
    \item \emph{Dimension} $ n=1 $: Im wesentlichen $ \R $ (nicht \hyperref[def:kompakt]{kompakt}) oder $ S^1 $ (kompakt).
    \item \emph{Dimension} $ n=2 $: Liste für \hyperref[def:zusammenhaengend]{zusammenhängende}, kompakte, ``orientierbare'', ``randlose'' Mannigfaltigkeiten:
    \begin{itemize}
      \item $ g = 0 $: $ S^2 $ \hyperref[bsp:einheitssphaere]{Einheitssphäre}
      \item $ g = 1 $: $ T^2 = S^1 \times S^1 $ Torus
      \item $ g = 2 $: Brezel
      \item \dots 
    \end{itemize}
    $ g $ ist das \term{Geschlecht}\label{def:geschlecht} der Mannigfaltigkeit.
    \item \emph{Dimension} $ n=3 $: Thurston's \term{Geometrisierungs-Vermutung}\label{theorem:geometrisierungsvermutung} ($ \sim $ 1978) \\
      Bewiesen von Perelman (2002), ein Milleniumsproblem.
    \item \emph{Dimension} $ n \geq 4 $: Allgemeine Klassifikation unmöglich, weil das Homöomorphieproblem hier nicht entscheidbar ist (Markov, 1960).
  \end{itemize}
\end{remark}

\section{Differenzierbare Mannigfaltigkeiten}

\begin{definition}[Kartenwechsel, differenzierbare Mannigfaltigkeit]
  Sei $ M $ \hyperref[def:topologischeMannigfaltigkeit]{topologische Mannigfaltigkeit}, $ p \in M $. Ein \term{Kartenwechsel}\label{def:kartenwechsel} ist ein \hyperref[def:homoeomorphismus]{Homöomorphismus}
  \begin{equation*}
    \psi \circ \phi^{-1}: \underbrace{\phi(D)}_{\subset \R^n} \to \underbrace{\psi(D)}_{\subset \R^n}\text{.}
  \end{equation*}
  Ein \hyperref[def:atlas]{Atlas} $ \mathcal{A} $ von $ M $ ist ein \term{$ C^\infty $-Atlas}\label{def:dbatlas}, falls alle möglichen Kartenwechsel $ C^\infty $-Abbildungen von $ \R^n $ sind, also alle partiellen Ableitungen existieren und stetig sind. \\
  Ein maximaler $ C^\infty $-Atlas heißt \term{$ C^\infty $-Struktur}\label{def:dbstruktur} auf der topologischen Mannigfaltigkeit $ M $. Eine $ C^\infty $-Mannigfaltigkeit ist eine topologische Mannigfaltigkeit mit einer $ C^\infty $-Struktur (auch \term{glatte}\label{def:glattemannigfaltigkeit} oder \term{differenzierbare Mannigfaltigkeit}\label{def:dbmannigfaltigkeit}).
\end{definition}

\begin{remark}
  \
  \begin{enumerate}
    \item Es gibt \hyperref[def:topologischeMannigfaltigkeit]{topologische Mannigfaltigkeiten} ohne \hyperref[def:dbstruktur]{differenzierbare Struktur}\footnote{Kerraire 1960}.
    \item Auf $ \R^n $, $ n \neq 4 $\footnote{Kirby, Friedman 1980}, existiert genau eine differenzierbare Struktur.
    \item Auf $ S^7 $ existieren $ 28 $ differenzierbare Strukturen\footnote{Milnor 1956}.
  \end{enumerate}
  \emph{Frage}: Wozu die Differenzierbarkeitsbedingung für \hyperref[def:kartenwechsel]{Kartenwechsel}? Beispielsweise für die Definition von differenzierbaren Abbildungen zwischen \hyperref[def:dbmannigfaltigkeit]{differenzierbaren Mannigfaltigkeiten}.
\end{remark}


\begin{definition}[Differenzierbarkeit]
  Seien $ M^m $, $ N^n $ \hyperref[def:dbmannigfaltigkeit]{differenzierbare Mannigfaltigkeiten} und $ F: M^m \to N^n $ stetig. $ F $ heißt \term{differenzierbar in $ p \in M $}\label{def:differenzierbar}, falls für \hyperref[def:karte]{Karten} $ (U, \phi) $ um $ p $ und $ (V, \psi) $ um $ F(p) $ gilt:
  \begin{equation*}
    \psi \circ F \circ \phi^{-1}: \underbrace{\phi(U)}_{\subset \R^m} \to \underbrace{\psi(V)}_{\subset \R^n}
  \end{equation*}
  ist $ C^\infty $-Abbildung in $ \phi(p) $. \\
  So kommt man von einem abstrakten $ F $ zwischen den Mannigfaltigkeiten zu einer konkreten Darstellung von $ F $. \\
  $ F $ heißt \term{differenzierbar} ($ C^\infty $), falls $ F $ differenzierbar ist für alle $ p \in M $.
  % TODO Abbildung 1
\end{definition}

\begin{remark}[Wohldefiniertheit der Differenzierbarkeit]
  Differenzierbarkeit in $ p $ ist wohldefiniert (also unabhängig von der Wahl der Karten)
  \begin{proof}
    \emph{Erster Test}: $ \psi \circ F \circ \phi^{-1} $, \emph{zweiter Test} $ \widetilde{\psi} \circ F \circ \widetilde{\phi}^{-1} $ \\
    Es gilt:
    \begin{align*}
      \psi \circ F \circ \phi^{-1} &= \psi \circ \underbrace{\widetilde{\psi}^{-1} \circ \widetilde{\psi}}_{\text{Id}_{\R^n}} \circ F \circ \underbrace{\widetilde{\phi}^{-1} \circ \widetilde{\phi}}_{\text{Id}_{\R^n}} \circ \phi^{-1} \\
        &= \underbrace{\left( \psi \circ \widetilde{\psi}^{-1} \right)}_{C^\infty} \circ \left( \widetilde{\psi} \circ F \circ \widetilde{\phi}^{-1} \right) \circ \underbrace{\left( \widetilde{\phi} \circ \phi^{-1} \right)}_{\text{Kartenwechsel}}
    \end{align*}
    Also: Abbildung in Test 1 ist $ C^\infty $ $ \Leftrightarrow $ Abbildung in Test 2 ist $ C^\infty $.
  \end{proof}
\end{remark}

\clearpage

\begin{remark}
  \
  \begin{itemize}
    \item $ N = \R $, $ F:M \to \R $ (differenzierbar) heißt \term{differenzierbare Funktion}.
    \item $ M = \R $ (oder $ I \subset \R $), $ F: I \to N $ heißt \term{differenzierbare Kuve}.
    \item Eine Abbildung $ F: M \to N $ zwischen differenzierbaren Mannigfaltigkeiten heißt \term{Diffeomorphismus}\label{def:diffeomorphismus}, falls $ F $ bijektiv und $ F $ und $ F^{-1} $ differenzierbar sind (also $ C^\infty $).
    \item Ein Homöomorphismus ist nicht unbedingt ein Diffeomorphismus. Beispielsweise $ \R $ mit Id als Karte, $ F : \R \to \R $, $ x \mapsto x^3 $ ist Homöomorphismus, aber kein Diffeomorphismus, da $ F^{-1} : x \mapsto \sqrt[3]{x} $ ist nicht $ C^\infty $.
    \item Die Menge der Diffeomorphismen einer differenzierbaren Mannigfaltigkeit ist eine Gruppe mit der Verkettung von Abbildungen.
  \end{itemize}
\end{remark}

\begin{example}
  \
  \begin{enumerate}
    \item $ U \subseteq \R^n $ offen (bzgl. Standard-Topologie). \\
      $ \phi_0 \coloneqq \text{Id}\mid_U $ mit zugehörigem maximalen Atlas definiert $ C^\infty $-Struktur auf $ U $, die kanonische differenzierbare Struktur.
    \item $ 2 $-dimensionale Mannigfaltigkeiten heißen auch \term{Flächen}\label{def:flaeche}, speziell \emph{regulär parametrisierte Flächen}\footnote{Gegenstand der klassischen Differentialgeometrie, siehe auch Kapitel 5}.
  \end{enumerate}
\end{example}

\begin{definition}[Reguläre Fläche]
  Eine Teilmenge $ S $ von $ \R^3 $ (mit Teilraum-Topologie von $ \R^3 $) heißt \term{reguläre Fläche}\label{def:regulaereFlaeche}, falls für jeden Punkt $ p \in S $ eine Umgebung $ V $ von $ p $ in $ \R^3 $ und eine Abbildung
  \begin{align*}
    F : \underset{\text{offen}}{U} \subset \R^2 &\to \underset{\text{offene TM von S}}{V \cap S} \subset \R^3 \\
      (u,v) &\mapsto (x(u,v),y(u,v),z(u,v))
  \end{align*}
  existiert, so dass gilt:
  % TODO Abbildung 2
  \begin{enumerate}
    \item $ F $ ist ein differenzierbar Homöomorphismus
    \item das Differential (Jacobi-Matrix) von $ F $,
    \begin{equation*}
       \text{d}F_q : \R^2 \supseteq T_qU \to T_{F(q)}\R^3 \cong \R^3
     \end{equation*} 
     ist \emph{injektiv} (d.h. Jacobi-Matrix hat Rang $ 2 $) für $ \forall q \in U $.
  \end{enumerate}
  $ F $ heißt \term{lokale Parametrisierung}\label{def:lokaleParametrisierung} von $ S $.
\end{definition}

\begin{example}[Rotationsfläche]
  Gegeben ist eine ebene Kurve $ c(v) = \left( r(v), 0, h(v) \right) $, $ v \in [a,b] $ mit $ r(v) > 0 $, $ c'(v) = \left( r'(v), 0, h'(v) \right) $ Tangentialvektor (mit $ C^\infty $-Funktionen $ r $, $ h $).\footnote{$ \Vert c'(v) \Vert \neq 0 \Leftrightarrow (r')^2 + (h')^2 \neq 0 $}
  % TODO Abbildung 3
  \begin{equation*}
    F(u,v) \coloneqq \begin{pmatrix}
      r(v)\cos u \\
      r(v)\sin u \\
      h(v)
    \end{pmatrix}
  \end{equation*}
  ist reguläre Fläche.\footnote{Übung!} \\
  \emph{Beispiel}: $ 2 $-Sphäre von Radius $ R $:
  \begin{equation*}
    (u,v) \mapsto \begin{pmatrix}
      R\cos v \cos u \\
      R\cos v\sin u \\
      R\sin v
    \end{pmatrix}\text{.}
  \end{equation*}
  Es gibt andere Parametrisierungen, beispielsweise
  \begin{equation*}
    (u,v) \mapsto \begin{pmatrix}
      u \\ v \\ \sqrt{R^2-u^2-v^2}
    \end{pmatrix}
  \end{equation*}
  % TODO Abildung 4
\end{example}

\begin{remark}[Geometrische Eigenschaften parametrisierungsunabhängig]
  \ \\
  Geometrische Eigenschaften sollten unabhängig sein von Parametrisierung. Das wird durch Eigenschaft 2 von regulären Flächen garantiert. Genauer gilt: Parameterwechsel sind differenzierbar ($ \leadsto $ reguläre Flächen sind differenzierbare $ 2 $-dimensionale Mannigfaltigkeiten mit $ F^{-1} $ (Umkehr-Abbildung der Parametrisierung) als Karten): \\
  Sei $ p \in S $ und $ F_1 : \R^2 \supseteq U \to S $, $ F_2: \R^2 \supseteq V \to S $ zwei Parametrisierungen, sodass $ p \in F_1(U) \cap F_2(V) \eqqcolon W $. \\
  % TODO Abbildung 5
  \emph{Behauptung}: Der Parameterwechsel 
  \begin{equation*}
    H \coloneqq F_1^{-1} \circ F_2 : \R^2 \supset F_2^{-1}(W) \to F_1^{-1}(W) \subset \R^2
  \end{equation*}
  ist Diffeomorphismus.
  \begin{proof}
    $ H $ ist Homöomorphismus, da $ F_1 $ und $ F_2 $ Homöomorphismen sind. \\
    \emph{Problem}: $ F_1^{-1} $ ist auf einer offenen Teilmenge von $ S $ definiert und und da weiß man nicht was \emph{differenzierbar} heißt. \\
    \emph{Ausweg}: Erweiterung von $ F $. Sei $ r \in F_2^{-1}(W) $ und $ q \coloneqq H(r) $. Da
      \begin{equation*}
        F_1(u,v) = \left( x(u,v),y(u,v),z(u,v) \right)
      \end{equation*}
    reguläre Parametrisierung ist können wir oBdA (erst Koordinatenachsen von $ \R^3 $ umbenennen) annehmen, dass
    \begin{equation*}
      \frac{J(x,y)}{J(u,v)}(q) \neq 0 \quad \text{(Jacobi-Determinante).}
    \end{equation*}
    \emph{Trick}: Erweitere $ F_1 $ zu Abbildung
    \begin{align*}
      \widetilde{F_1} : U \times \R &\to \R^3 \\
        \widetilde{F_1}(u,v,t) &\coloneqq \left( x(u,v),y(u,v),z(u,v)+\bm{t} \right)\text{.}
    \end{align*}
    $ \widetilde{F_1} $ ist differenzierbar und $ \widetilde{F_1}|_{U \times \{ 0 \}} = F_1 $. \\
    Die Jacobi-Determinante von $ \widetilde{F_1} $ in $ (q,0) $,
    \begin{equation*}
      \det \begin{pmatrix}
        \frac{\text{d}x}{\text{d}u} & \frac{\text{d}x}{\text{d}v} & 0 \\
        \frac{\text{d}y}{\text{d}u} & \frac{\text{d}y}{\text{d}v} & 0 \\
        \frac{\text{d}z}{\text{d}u} & \frac{\text{d}z}{\text{d}v} & 1
      \end{pmatrix}(q,0) = \det \begin{pmatrix}
        \frac{\text{d}x}{\text{d}u} & \frac{\text{d}x}{\text{d}v} \\
        \frac{\text{d}y}{\text{d}u} & \frac{\text{d}y}{\text{d}v}
      \end{pmatrix}(q) \neq 0\text{.}
    \end{equation*}
    Nach dem Umkehrsatz (Analysis II) existiert eine Umgebung $ A $ von $ \widetilde{F_1}(q,0) = F_1(q) $ in $ \R^3 $ sodass $ \widetilde{F_1}^{-1} $ auf $ A $ existiert und differenzierbar ($ C^\infty $) ist. Da $ F_2 $ stetig ist existiert Umgebung $ B $ von $ v $ in $ V $, sodass $ F_2(B) \subset A $. Und nun ist $ H|_B = \widetilde{F_1}^{-1} \circ F_2|_B $ ist Verkettung von differenzierbaren Abbildungen, also differenzierbar in $ r $ und da $ r $ beliebig ist ist $ H $ differenzierbar auf $ F_2^{-1}(W) $. 
  \end{proof}
\end{remark}

\begin{example}[Weitere Beispiele von differenzierbaren Mannigfaltigkeiten]
  \
  \begin{enumerate}
    \item \textbf{n-Sphäre} von Radius $ R $ (und Zentrum $ 0 $):
      \begin{equation*}
        S_R^n \coloneqq \{ x \in \R^{n+1} : \Vert x \Vert = R \}\text{.}
      \end{equation*}
      Karten via stereographischer Projektion.
      \begin{align*}
        &N \coloneqq (0, \dots, 0, R), \quad &S \coloneqq (0, \dots, 0, -R) \\
        &U_1 \coloneqq S_R^n \setminus \{ N \}, \quad &U_2 \coloneqq S_R^n \setminus \{ S \}
      \end{align*}
      Stereographische Projektion bzgl $ N $:
      \begin{equation*}
        \phi_1 : U_1 \to \R^n, \ p = (p_1, \dots, p_{n+1}) \mapsto \left( x_1(p), \dots, x_n(p) \right), \ x_i(p) \coloneqq \frac{Rp_i}{R-p_{i+1}}
      \end{equation*}
      Stereographische Projektion bzgl. $ S $:\footnote{\textbf{Übung}: $ \phi_1 $ und $ \phi_2 $ sind Homöomorphismen.}
      \begin{equation*}
        \phi_1 : U_2 \to \R^n, \ p = (p_1, \dots, p_{n+1}) \mapsto \left( x_1(p), \dots, x_n(p) \right), \ x_i(p) \coloneqq \frac{Rp_i}{R-p_{i+1}}
      \end{equation*}
      Kartenwechsel:
      \begin{equation*}
        \phi_2 \circ \phi_1^{-1} : \R^n \setminus \{ 0 \} \to \R^n \setminus \{ 0 \}, \ \phi_2 \circ \phi_1^{-1}(x) = \frac{x}{\Vert x \Vert}R^2
      \end{equation*}
      ist $ C^\infty $. \\
      $ \Rightarrow \mathcal{A} \coloneqq \{ (U_1, \phi_1), (U_2, \phi_2) \} $ ist ein differenzierbarer Atlas für $ S_R^n $. \\
      $ \leadsto $ max. Atlas aller mit $ \mathcal{A} $ verträglichen Karten (also allen $ (U, \phi) $) mit $ \phi \circ \psi^{-1} $ ist $ C^\infty $ für $ \psi $ aus $ \mathcal{A} $ sofern Verkettung definiert ist) definiert differenzierbare Struktur auf $ S_R^n $, also ist $ S_R^n $ eine $ C^\infty $-Mannigfaltigkeit mit Dimension $ n $.
    \item $ n $-dimensionaler reell projektiver Raum
      \begin{equation*}
        P^n\R \coloneqq \{ \text{1-dim. UVR von } \R^{n+1} \} \equiv \left( \R^{n+1} \setminus \{ 0 \} \right) /\sim
      \end{equation*}
      mit $ x \sim y \overset{\text{Def.}}{\Leftrightarrow} \ \exists \ \R \ni \lambda \neq 0 : x = \lambda y $ ($ 1 $-dimensionaler UVR = Äquivalenzklasse) $ \equiv S^n/\sim $ mit $ x \sim y \overset{\text{Def.}}{\Leftrightarrow} x = -y $. \\
      Wir sehen: \\
      \emph{1. Definition}: Eindimensionale Untervektorräume \\
      \emph{2. Definition}: Äquivalenzklassen in $ \R^{n+1} \setminus \{ 0 \} $ \\
      \emph{3. Definition}: Äquivalenzklassen in $ S^n $ \\
      Es ist leicht zu sehen, dass diese Definitionen äquivalent sind. \\
      Aus der 3. Definition sieht man
      \begin{equation*}
        P^n\R = S^n/\sim
      \end{equation*}
      ist kompakt als Quotientenraum von $ S^n $ (Quotiententopologie $ X \overset{\pi}{\to} Y = X/\sim $ mit topologischem Raum $ X $ und Quotiententopologie: $ U $ offen in $ Y \Leftrightarrow \pi^{-1}(U) $ ist offen in $ X $). Diese Abbildung ist stetig, und ein stetiges Bild von einer kompakten Menge ist wieder kompapkt. \\
      \textbf{Karten}:
      \begin{align*}
        \tilde{U_i} &\coloneqq \{ x \in S^n : x_i \neq 0 \}, \quad i = 1, \dots, n+1 \\
        U_i &\coloneqq \pi(\tilde{U_i}) \text{ mit } \pi: S^n \to S^n/\sim = P^n\R\text{.}
      \end{align*}
      Projektion:
      \begin{equation*}
        \phi_i : U_i \to \R^n, \quad \phi_i([x]) \coloneqq \left( \frac{x_i}{x_i}, \dots, \frac{x_{i-1}}{i}, \frac{x_{i+1}}{i}, \dots, \frac{x_n}{x_i} \right)
      \end{equation*}
      sind Homöomorphismen.\footnote{\textbf{Übung}: Kartenwechsel $ \phi_i \circ \phi_j^{-1} $ sind $ C^\infty $.}
  \end{enumerate}
\end{example}

\begin{remark}
  Man kann zeigen: $ P^n\R $ ist hausdorffsch und hat eine abzählbare Basis der Topologie. Also ist $ P^n\R $ eine $ n $-dimensionale $ C^\infty $-Mannigfaltigkeit. \\
  % TODO Abbildung 1
  \emph{Analog}: $ P^n\C \coloneqq \{ \text{komplexe $ 1 $-dim. UVR von } C^{n+1} \} $ ist kompakte $ 2n $-dimensionale $ C^\infty $-Mannigfaltigkeit.
\end{remark}

\begin{example}[Produkt-Mannigfaltigkeiten]
  Für $ M^m $ und $ N^n $ $ m $- bzw. $ n $-dimensionale differenzierbare Mannigfaltigkeit ist die \term{Produkt-Mannigfaltigkeit}\label{def:produktmannigfaltigkeit} $ M \times N $ eine $ (m+n) $-dimensionale $ C^\infty $-Mannigfaltigkeit.\footnote{\textbf{Übung}!}
\end{example}

\begin{excourse}[Lie-Gruppen]
  Eine \term{Lie-Gruppe}\label{def:liegruppe} ist eine Gruppe mit einer $ C^\infty $-Mannigfaltigkeitstruktur, so dass die Abbildung
  \begin{equation*}
    G \times G \to G, \quad (g,h) \mapsto gh^{-1}
  \end{equation*}
  $ C^\infty $ ist.
\end{excourse}

\begin{example}[zu Lie-Gruppen]
  \
  \begin{itemize}
    \item $ (\Z, +) $ ist eine $ 0 $-dimensionale Lie-Gruppe.
    \item $ SO(2) = \left\{ \begin{pmatrix}
      \cos \theta & -\sin \theta \\
      \sin \theta & \cos \theta
    \end{pmatrix} : \theta \in [0, 2\pi] \right\} \underset{\text{homö}}{\cong} S^1 $ ist kompakte $ 1 $-dimensionale $ C^\infty $-Mannigfaltigkeit und Lie-Gruppe.\footnote{\textbf{Übung}: Wieso?} 
    \item $ SU(2) \coloneqq \left\{ \begin{pmatrix}
      \alpha & \beta \\
      -\overline{\beta} & \overline{\alpha}
    \end{pmatrix} : \alpha, \beta \in C, \ \alpha\overline{\alpha}+\beta\overline{\beta} = 1 \right\} \underset{\text{homö}}{\cong} S^3 $ ist kompakte $ 3 $-dimensionale $ C^\infty $-Mannigfaltigkeit.\footnote{$ 1 = \alpha\overline{\alpha}+\beta\overline{\beta} = x_1^2 + x_2^2 + x_3^2 + x_4^2 $ mit $ \alpha = x_1+\i x_2 $ und $ \beta = x_3 + \i x_4 $.}
    \item $ GL(n, \R) $ (offene) Untermannigfaltigkeit von $ \R^{n^2} \leadsto n^2 $-dimensionale $ C^\infty $-Mannigfaltigkeit.
  \end{itemize}
\end{example}

\begin{remark}[Fakt von Cartan]
  Abgeschlossene Untergruppen von Lie-Gruppen sind Lie-Gruppen sind auch Lie-Gruppen.
\end{remark}

\begin{example}[Fakt von Cartan benutzen]
  \
  \begin{align*}
    SO(n) &= \{ A \in GL(n, \R) : AA^\top = E, \ \det A = 1 \} \text{ und} \\
    SL(n, \R) &= \{ A \in GL(n, \R) : \det A = 1 \}
  \end{align*}
  sind Lie-Gruppen: Benutze, dass
  \begin{equation*}
    A = \{ x \in X : f(x) = g(x) \} \text{ und } X \text{ ist hausdorffsch}
  \end{equation*}
  $ \Rightarrow A \text{ abgeschlossen, } f, g \text{ stetige Abbildungen} $
\end{example}

\section{Simplizialkomplexe}

Simplizialkomplexe sind Objekte der algebraischen Topologie. Mittels Kombinatorik sollen topologische Invarianten bestimmt werden.

\begin{definition}[Simplex]
  Ein $ k $-dimensionales \term{Simplex}\label{def:simplex} im $ \R^n $ ist die konvexe Hülle von $ k+1 $ Punkten $ v_0, \dots, v_k $ in allgemeiner Lage:
  \begin{equation*}
    s(v_0, \dots, v_k) \coloneqq \left\{ \sum_{i=0}^n \lambda_i v_i : \forall \lambda_i \geq 0, \ \sum_{i = 0}^k \lambda_i = 1 \right\}
  \end{equation*}
  für $ v_0-v_1, \dots, v_0-v_k $ linear unabhängig.
\end{definition}

\begin{example}[Einfache Simplices]
  \
  \begin{itemize}
    \item \textbf{0-Simplex}: $ v_0 $ (Punkt) 
    \item \textbf{1-Simplex}: $ v_0 $ --- $ v_1 $ (Strecke, $ s(v_0, v_1) = \{ \lambda v_0 + (1-\lambda)v_1 : 0 \leq \lambda \leq 1 \} $)
    \item \textbf{2-Simplex}: $ \triangle v_0, v_1, v_2 $ (Dreicksfläche)
    \item \textbf{3-Simplex}: $ v_0, v_1, v_2, v_3 $ ((volles) Tetraeder)
  \end{itemize}
\end{example}

\begin{definition}[Teilsimplex, Seite]
  Die konvexe Hülle einer Teilmenge von $ \{ v_0, \dots, v_k \} $ heißt \term{Teilsimplex}\label{def:seite} oder \term{Seite} von $ s(v_0, \dots, v_k) $.
  % TODO Abbildung 2
\end{definition}

\begin{definition}[Simplizialkomplex]
  Eine endliche Menge $ K $ von Simplices in $ \R^n $ heißt \term{Simplizialkomplex}\label{def:simplizialkomplex}, wenn gilt:
  \begin{enumerate}
    \item Mit jedem seiner Simplices enthält $ K $ auch dessen sämtliche Teilsimplices.
    \item Der Durchschnitt von je zwei Simplices ist entweder leer oder ein gemeinsamer Teilsimplex. 
  \end{enumerate}
  % TODO Abbildung 3
\end{definition}

\begin{definition}[Geometrische Realisierung]
  \
  \begin{equation*}
    \vert K \vert \coloneqq \bigcup_{s \in K} s \subset \R^n
  \end{equation*}
  mit Teilraumtopologie von $ \R^n $ heißt der dem Simplizialkomplex $ K $ zugrunde liegende topologische Raum. \\
  \emph{Achtung}: Verschiedene Simplizialkomplexe $ K $, $ K' $ können das gleiche $ \vert K \vert = \vert K' \vert $ haben.
  % TODO Abbildung 1
\end{definition}

\begin{remark}[Vorteil von Simplizialkomplexen]
  Kennt man von einem (endlichen) Simplizialkomplex die \term{wesentlichen Simplices}\label{def:wesentlicheSimplices} (also solche, die nicht Seiten von anderen sind) in jeder Dimension und ihre \term{Inzidenzen}\label{def:inzidenzen} (also welche Ecken sie gemeinsam haben), so kennt man $ \vert K \vert $ (bis auf Homöomorphie).
  \begin{proof}[Konstruktionsidee von $ \vert K \vert $ aus diesen Daten]
    \
    \begin{enumerate}
      \item Wähle in jeder Dimension einen \emph{Standard-Simplex} $ \Delta_k \coloneqq s(\underbrace{e_1, \dots, e_{k+1}}_{\text{Std.-Basis-Vek.}}) $ 
      \item Bilde disjunkte Vereinigung von solchen $ \Delta_k $ in jeder Dimension $ k $ soviele wie es wesentliche $ k $-Simplices gibt:
        \begin{equation*}
          X \coloneqq \underbrace{\Delta_0 \cup \dots \cup \Delta_0}_{\text{\# wesentliche $ 0 $-Simp.}} \cup \dots \cup \underbrace{\Delta_n \cup \dots \cup \Delta_n}_{\text{\# wesentliche $ n $-Simp.}}
        \end{equation*}
      \item Identifiziere Inzidenzen (via Äquivalenzrelation) gemäß Inzidenz-Angaben für Ecken
    \end{enumerate}
    Diese drei Schritte liefern dann eine stetige Bijektion des (kompakten) Quotientenraumes $ X/\sim $ auf Hausdorff-Raum $ \vert K \vert $, also ein Homöomorphismus.
  \end{proof}
\end{remark}

\begin{definition}[Dimension]
  Die \term{Dimension} eines Simplizialkomplexes $ K $ ist die maximale Dimension seiner Simplices.
\end{definition}

\begin{remark}[Spezialfall --- Graph]
  Ein \term{endlicher Graph}\label{def:graph} ist ein endlicher, $ 0 $- oder $ 1 $-dimensionaler Simplizialkomplex,\footnote{Aufgrund der Eindimensionalität haben beispielsweise die Dreiecke in einem Graph keine Füllung!} gebaut aus $ 1 $-dimensionalen (\emph{Kanten}) und $ 0 $-dimensionalen (\emph{Ecken}) Simplices. \\
  Ein Graph $ G $ heißt \term{zusammenhängend}\label{def:zusammenhaengend}, falls zu je zwei Ecken $ p, p' \in G $ eine Folge $ p = p_0, p_1, \dots, p_n = p' $ paarweise verschiedener Ecken von $ G $ existiert, sodass $ p_{i-1} $ und $ p_i $ durch eine Kante verbunden sind. \\
  Ein \term{Baum}\label{def:baum} ist ein zusammenhängender Graph $ T $, so dass für jedes $ 1 $-Simplex (\emph{Kante}) $ s \in T $ gilt: $ \vert T \vert \setminus \mathring{s} $ ist nicht zusammenhängend (mit $ \mathring{s} = $ \emph{offener} $ 1 $-Simplex, also Kante ohne Endpunkte).
\end{remark}

\begin{definition}[Euler-Charakteristik]
  Sei $ G $ ein endlicher Graph,
  \begin{align*}
    \alpha_0 &\coloneqq \text{ Anzahl Ecken in } G\text{,} \\
    \alpha_1 &\coloneqq \text{ Anzahl Kanten in } \text{G.}
  \end{align*}
  Die \term{Euler-Charakteristik}\label{def:eulerCharakteristik} von $ G $ ist
  \begin{equation*}
    \chi(G) \coloneqq \alpha_0 - \alpha_1
  \end{equation*}
  \emph{Bemerkung}: $ \chi(G) $ ist invariant unter Unterteilung (also dem Hinzufügen von neuen Ecken auf einer Kante).
\end{definition}

\begin{theorem}[$ \chi $ von Bäumen]
  Sei $ T $ ein (endlicher) Baum. Dann gilt $ \chi(T) = 1 $.
  \begin{proof}
    Induktion nach $ \alpha_0 = $ Anzahl Ecken.
    \begin{itemize}
      \item $ n = 1 $. Dann ist $ G $ ein Punkt, $ \alpha_0 = 1 $, $ \alpha_1 = 0 $, $ \chi(T) = \alpha_0 - \alpha_1 = 1 \quad \checkmark $ 
      \item $ n = 2 $. Dann ist $ G $ eine Kante mit Endpunkten, $ \alpha_0 = 2 $, $ \alpha_1 = 1 $, $ \chi(T) = 1 \quad \checkmark $
      \item \textbf{Induktionsannahme}: Satz gilt für alle Bäume mit $ n $ Ecken.
      \item \textbf{Induktionsschritt}: $ \chi(T) = 1 $ für Bäume mit $ n+1 $ Ecken. \\
        Sei $ T $ ein Baum mit $ n+1 $ Ecken und $ v_0 $ ein \term{Ende}\label{def:blatt} von $ T $ (also eine Ecke die zu genau einer Kante gehört). Ein solches Ende existiert.\footnote{vgl. Übung} \\
        Sei $ \vert T_1 \vert \coloneqq \vert T \vert \setminus \{ \mathring{s_1} \cup v_0 \} $. $ T_1 $ ist wieder ein Baum, sonst existiert $ s_2 $ sodass $ T_1 \setminus \{ \mathring{s_2} \} $ zusammenhängend ist, also auch $ T \setminus \{ \mathring{s_2} \} $ zusammenhängend $ \lightning $. \\
        $ T_1 $ hat $ n $ Ecken, also nach IV: $ \chi(T_1) = 1 $. \\
        Da $ \alpha_0(T) = \alpha_0(T_1) + 1 $ und $ \alpha_1(T) = \alpha_1(T_1) + 1 $ ist $ \chi(T_1) = 1 $. \qed
    \end{itemize}
  \end{proof}
\end{theorem}

\begin{theorem}[$ \chi $ von zusammenhängenden Graphen]
  Sei $ G $ ein zusammenhängender, endlicher Graph. Sei $ n $ die Anzahl von offenen $ 1 $-Simplices (\emph{Kanten}), die man aus $ G $ entfernen kann, sodass $ G $ zusammenhängend bleibt. Dann ist $ \chi(G) = 1 - n $.\footnote{Die Aussage aus dem vorhergehenden Satz folgt aus diesem direkt.}
  \begin{proof}
    Ist $ G $ ein Baum, so ist $ n = 0 $ und die Behauptung gilt. \\
    Ist $ G $ \underline{kein} Baum, so existiert ein offenes $ 1 $-Simplex $ \mathring{s_1} $, sodass $ \vert G_1 \vert = \vert G \vert \setminus \{ \mathring{s_1} \} $ zusammenhängend ist. Ist $ G_1 $ ein Baum, so hält man an. Ist $ G_1 $ kein Baum, so entfernt man eine Kante $ \mathring{s_2} $ usw. \\
    $ G $ hat endlich viele Kanten, also existiert ein max. $ n $, so dass $ \vert G \vert \setminus \{ \mathring{s_1} \cup \cdots \cup \mathring{s_n} \} $ ein Baum ist. \\
    Es gilt dann $ \chi(G) = \chi(T) - n = 1-n $. \qed
  \end{proof}
  \emph{Bemerkung}: Das Komplement $ T $ aller offenen Kanten die man aus $ G $ entfernen kann (wie im Beweis) ist ein sog. \term{spannender Baum}\label{def:spannenderBaum} für $ G $, der alle Ecken in $ G $ enthält (nicht eindeutig).
\end{theorem}

\begin{definition}[Ebene und planare Graphen]
  Ein Graph heißt \term{eben}\label{def:eben}, falls er durch Punkte und Geradenstücke in der Ebene (also $ \R^2 $) realisiert ist, so dass sich die Kanten nicht schneiden (außer in den Ecken). \\
  Ein (abstrakter) Graph (also gegeben durch Ecken-Mengen und Inzidenzen) heißt \term{planar}\label{def:planar}, falls er \emph{isomorph} zu einem ebenen Graphen ist.  
\end{definition}

\begin{example}
  \
  \begin{enumerate}
    \item $ K_4 = $ vollständiger Graph mit $ 4 $ Ecken (d.h. alle Ecken-Paare sind durch Kanten verbunden). Zeichnet man diesen Graphen als Quadrat, so ist dieser nicht eben. Man kann aber $ K_4 $ so zeichnen, dass der Graph eben ist. Also ist $ K_4 $ planar.
    \item $ K_5 = $ vollständiger Graph mit $ 5 $ Ecken. Dieser Graph ist nicht isomorph zu einem ebenen Graphen, also ist $ K_5 $ nicht planar.
  \end{enumerate}
\end{example}

\begin{definition}[Seiten]
  Die \term{Seiten}\label{def:seiten} eines ebenen Graphen $ G $ sind die Zusammenhangskomponenten von $ \R^2 \setminus G $.
\end{definition}

\begin{theorem}[Euler-Formel]
  Für einen zusammenhängenden, ebenen Graphen $ G $ gilt:
  \begin{equation*}
    \chi(G) \coloneqq e(G) - k(G) + s(G) = 2\text{,}
  \end{equation*}
  wobei $ e(G) $ die Anzahl Ecken von $ G $, $ k(G) $ die Anzahl Kanten von $ G $ und $ s(G) $ die Anzahl Seiten von $ G $ ist. \\
  $ \chi(G) $ ist die \term{Euler-Charakteristik}\label{def:eulercharakteristik} von $ G $.
  \begin{proof}
    Sei $ T $ ein \term{aufspannender Baum}\label{def:aufspannenderBaum} für $ G $ (also ein Baum der alle Ecken von $ G $ enthält). Dann gilt $ e(T)-k(T) = 1 $ und $ s(T) = 1 $. Also gilt die Behauptung für $ T $. \\
    $ G $ erhält man aus $ T $ durch Hinzufügen von Kanten. Für jede neue Kante entsteht auch eine neue Seite, welche sich in der Summe aus der Behauptung aufheben. Also
    \begin{equation*}
      \chi(G) = e(G)-k(G)+s(G) = 2\text{.}
    \end{equation*}\qed
  \end{proof}
\end{theorem}

\begin{definition}[Polyeder]
  Eine Teilmenge $ P $ von $ \R^3 $ heißt \term{(konvexes) Polyeder}\label{def:polyeder}, falls
  \begin{enumerate}
    \item $ P $ ist Durchschnitt von endlich vielen \term{affinen Halbräumen}\label{def:affinerHalbraum} von $ \R^3 $ (d.h. gegeben durch Ungleichungen $ a_ix+b_iy+c_iz \geq d_i $, $ i = 1, \dots, k $)
    \item $ P $ ist beschränkt und nicht in einer Ebene enthalten.
  \end{enumerate}
  Der \term{Rand}\label{def:rand} von $ P $ besteht dann aus Seiten(flächen), Kanten und Ecken (gegeben als $ 2 $-dimensionale, $ 1 $-dimensionale und $ 0 $-dimensionale Schnitte von Ebenen).
\end{definition}

\begin{remark}[Bezug von Polyedern zu Graphen]
  Das \term{1-Skelett}\label{def:skelett} von $ P $ (also die Menge der Ecken und Kanten) von $ P $ ist ein Graph in $ \R^3 $. \\
  Man kann zeigen (Resultat der konvexen Geometrie): durch Zentralprojektion von einem Punkt nahe bei einem ``Seitenmittelpunkt'' auf eine geeignete Ebene wird das $ 1 $-Skelett $ p^{(1)} $ von $ P $ auf einen \emph{ebenen} Graphen $ G_p $ abgebildet (sog. \term{Schlegel-Diagramm}\label{def:schlegelDiagramm}). Es gilt dann: $ s(P) = s(G_p) $, $ k(P) = k(G_p) $, $ e(P) = e(G_p) $.
\end{remark}

\begin{deduction}[Eulersche Polyeder-Formel]
  \begin{equation*}
    e(P) - k(P) + s(P) = 2\text{.}
  \end{equation*}
\end{deduction}

\begin{definition}[Regulärer Polyeder]
  Ein Polyeder in $ \R^3 $ heißt \term{regulär}\label{def:regulaererPolyeder}, falls alle Seitenflächen kongruente reguläre $ n $-Ecke (d.h. sie haben gleich lange Kanten) sind und in jeder Ecke $ m $ solche $ n $-Ecke zusammentreffen (insbesondere gehen von jeder Ecke $ m $ Kanten aus).
\end{definition}

\begin{theorem}[Platonische Körper]
  Es gibt genau $ 5 $ reguläre Polyeder in $ \R^3 $:
  \begin{align*}
    (m,n) &= (3,3) \quad \text{Tetraeder}  \\
     &= (3,4) \quad \text{Würfel} \\
     &= (4,3) \quad \text{Oktaeder} \\
     &= (3,5) \quad \text{Dodekaeder} \\
     &= (5,3) \quad \text{Ikosaeder}
  \end{align*}
  \begin{proof}
    \
    \begin{itemize}
      \item \textbf{Existenz}: Explizite Konstruktion, siehe Euklid (oder Tutorium (oder basteln (oder Google))).
      \item \textbf{Vollständigkeit}: Sei $ s = $ Anzahl an Seitenflächen. Dann gilt: $ s*n = 2k $, ebenso $ m*e = 2k $ und damit
        \begin{equation*}
          n*s = 2k = m*e \Rightarrow k = \frac{me}{2} \quad s = \frac{me}{n}
        \end{equation*}
        Euler-Polyeder-Formel für $ P $ bzw. $ G_p $ ergibt:
        \begin{equation*}
          2 = e-k+s = e - \frac{me}{2} + \frac{me}{n} \Leftrightarrow 4n = e\left( 2n - nm + 2m \right)\text{.}
        \end{equation*}
        Da $ n > 0 $ und $ e > 0 $ folgt:
        \begin{equation*}
          2n - nm + 2m > 0 \Leftrightarrow nm - 2n - 2m + 4 < 4 \Leftrightarrow (n-2)(m-2) < 4\text{.}
        \end{equation*}
        Man sieht, dass es nur obenstehende Möglichkeiten gibt. \qed
    \end{itemize}
  \end{proof}
\end{theorem}

\begin{definition}[Euler-Charakteristik von Simplizialkomplexen]
  Sei $ K $ ein Simplizialkomplex. Dann ist die \term{Euler-Charakteristik}[def:eulercharakteristikSimplizialkomplex] von $ K $:
  \begin{equation*}
    \chi(K) \coloneqq \alpha_0-\alpha_1+\alpha_2 \mp \dots \pm \alpha_k = \sum_{i = 0}^k (-1)^i\alpha_i\text{,}
  \end{equation*}
  wobei $ \alpha_i = $ Anzahl von $ i $-Simplices in $ K $. \\
  Die sogenannten ``\term{Betti-Zahlen}''\label{def:bettiZahlen} lassen sich berechnen mit Methoden aus der algebraischen Topologie (als Dimension von gewissen Vektorräumen, die man zu $ K $ konstruiert). \\
  Man zeigt: $ \chi(K) $ ist eine topologische Invariante, also 
  \begin{equation*}
    \vert K \vert \underset{\text{homö}}{\cong} \vert \widetilde{K} \vert \Rightarrow \chi(K) = \chi(\widetilde{K})\text{.}
  \end{equation*}
  Ein topologischer Raum $ X $ heißt \term{triangulierbar}\label{def:triangulierbar}, falls ein (endlicher) Simplizialkomplex $ K $ existiert und ein Homöomorphismus $ \vert K \vert \overset{\sim}{\to} X $. \\
  Ist $ X $ (via $ K $) triangulierbar, so definiert man $ \chi(X) \coloneqq \chi(K) $ (und zeigt, dass $ \chi(X) $ unabhängig von der gewählten Triangulierung ist). \\
  Nun ist $ \chi(S^2) = \chi(\text{Tetraeder}) = 2 $ und jeder (reguläre) Polyeder homöomorph zu $ S^2 $, also $ \chi(P) = \chi(S^2) = 2 $.
\end{definition}

% TODO BSP1

\section{Spezielle Konstruktion von Quotientenräumen (``Verkleben'')}

\begin{definition}[Verklebung]
  $ X $ und $ Y $ seien topologische Räume, $ A \subset X $ ein Teilraum und $ f: A \to Y $ eine Abbildung (nicht notwendigerweise stetig). Sei $ X \cupdot Y $ die disjunkte Vereinigung. Definiere eine Äquivalenzrelation auf $ X \cupdot Y $ via $ f $ wie folgt:
  \begin{equation*}
    x \sim x' \overset{\text{Def}}{\Leftrightarrow} \begin{cases}
      &x = x' \\
      \text{oder } &f(x) = x' \quad (x \in A) \\
      \text{oder } &f(x') = x \quad (x' \in A) \\
      \text{oder } &f(x) = f(x') \quad (x, x' \in A)
    \end{cases}
  \end{equation*}
  Das ist eine Äquivalenzrelation. \\
  Der Quotientenraum $ X \cup_f Y \coloneqq X \cupdot Y /_\sim $ heißt \term{Verklebung}\label{def:verklebung} von $ Y $ an $ Y $ via $ f $.
\end{definition}

\begin{example} \
  \begin{enumerate}
    \item $ X = Y = S^1 $, $ A = \{ x_0 \} $, $ f(x_0) \coloneqq x_0 $
      % TODO BSP2
    \item $ X = [0,1] $, $ Y = [2,5] $, $ A = \{ 0,1 \} \subset X $, $ f(0) = 2 $, $ f(1) = 3 $
      % TODO BSP3
    \item \emph{Zusammenhängende Summe} von $ 2 $-Mannigfaltigkeiten $ M_1 $ und $ M_2 $.
      % TODO BSP4 
      \\
      Konstruktion:
      \begin{enumerate}
        \item Entferne geeignet kleine abgeschlossene ``Kreisscheiben'' von $ p_1 \in M_1 $ und $ p_2 \in M_2 $ mit Rändern $ \delta B_1 $ und $ \delta B_2 $ homöomorph zu $ S^1 $.
        \item Wähle Homöomorphismus $ f: \delta B_1 \to \delta B_2 $.
        \item Verklebe $ M_1 $ und $ M_2 $ mittels $ f : M_1 \cup_f M_2 \eqqcolon M_1 \# M_2 $
        % TODO BSP5
      \end{enumerate}
      Alle kompakten geschlossenen Flächen kann man aus $ S^2 $ konstruieren durch Verkleben Tori.
  \end{enumerate}
\end{example}

\begin{remark}[Selbstverklebungen]
  ``\term{Selbst-Verklebungen}''\label{def:selbstverklebung} sind analog definiert: \\
  $ X = $ topologischer Raum, $ A \subset X $ Teilraum, $ f : A \to X $, $ X_f \coloneqq X/_\sim $ mit Äquivalenzrelation wie oben.
\end{remark}

\begin{example}
  \
  \begin{enumerate}
    \item $ X = [0,1] \times [0,1] $ = Einheitsquadrat,
      \begin{equation*}
        A \subset \delta X = \underbrace{\left( \{ 0 \} \times [0,1] \right)}_{\eqqcolon A_1} \cup \underbrace{\left( \{ 1 \} \times [0,1] \right)}_{\eqqcolon B_2} \cup \underbrace{\left( [0,1] \times \{ 0 \} \right)}_{\eqqcolon A_2} \cup \underbrace{\left( [0,1] \times \{ 1 \} \right)}_{\eqqcolon B_2}\text{,}
      \end{equation*}
      $ A \coloneqq A_1 \cup A_2 $,
      \begin{align*}
        f: &A_1 \to B_1, \quad (0,t) \mapsto (1,t) \\
         &A_2 \to B_2, \quad (t,0) \mapsto (t,1)
      \end{align*}
      % TODO BSP6
      Man erhält letztendlich einen Torus.
    \item \emph{Möbiusband}: $ X = [0,1] \times [0,1] $, $ A = A_1 $, $ f: A_1 \ni (0,1) \mapsto (1,1-t) \in B_1 $
    % TODO BSP7
    \item \emph{Projektive Ebene}: $ P^2\R $ entsteht durch Verkleben einer Kreisscheibe und eines Möbiusbandes längs der Ränder.
    % TODO BSP8
    % TODO BSP9 --- Kleinsche Flasche
  \end{enumerate}
\end{example}
