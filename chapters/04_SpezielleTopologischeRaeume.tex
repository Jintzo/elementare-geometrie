\chapter{Spezielle Klassen von topologischen Räumen}

\section{Übersicht}

Folgende spezielle Klassen sollen diskutiert werden:
\begin{itemize}
  \item \hyperref[def:metrischerRaum]{metrische Räume} $ \leadsto $ metrische Geometrie
  \item \hyperref[def:topologischeMannigfaltigkeit]{Mannigfaltigkeiten} (Grundobjekte in Differenzialgeometrie, Physik,\dots)
  \item \hyperref[def:polyeder]{Polyeder}, \hyperref[def:simplizialkomplex]{Simplizialkomplexe} (Kombinatorik, algebraische Topologie)
  \item \hyperref[def:bahnenraum]{Bahnen-Räume} von \hyperref[def:gruppenaktion]{Gruppenaktionen} (geometrische Gruppentheorie)
\end{itemize}

\section{Topologische Mannigfaltigkeiten}

\begin{definition}[Topologische Mannigfaltigkeit]
  \label{def:topologischeMannigfaltigkeit}
  Eine \term{topologische Mannigfaltigkeit} ist ein \hyperref[def:topologie]{topologischer Raum} $ M $ mit folgenden Eigenschaften:
  \begin{enumerate}
    \item $ M $ ist \term{lokal euklidisch}\label{def:lokalEuklidisch}, d.h. $ \forall p \in M \ \exists $ offene Umgebung $ U $ von $ p $ und ein \hyperref[def:homoeomorphismus]{Homöomorphismus} $ \varphi: U \to \varphi(U) \subset \R^n $ mit festem $ n $. Das Paar $ (\varphi, U) $ heißt \term{Karte}\label{def:karte}\footnote{Eine mathematische Karte ist einer echten Karte ähnlich. Man nehme einen Punkt, zum Beispiel Karlsruhe, und beschreibt die Umgebung von Karlsruhe in Form einer Karte auf einer DIN A4-Karte. Das ist natürlich nicht bijektiv, aber man versucht es möglichst bijektiv zu machen.} und $ \mathcal{A} = \left\{ (\varphi_a, U_\alpha) : \alpha \in A \right\} $ mit $ \bigcup_{\alpha \in A}U_\alpha = M $ heißt \term{Atlas}\label{def:atlas}.
    \item $ M $ ist \hyperref[def:hausdorffsch]{hausdorffsch} und besitzt abzählbare Basis der Topologie.
  \end{enumerate}
  \emph{Bemerkung}:
  \begin{itemize}
    \item Die zweite Eigenschaft ist ``technisch'' und garantiert , dass eine ``Zerlegung der Eins'' existiert (braucht man z.B. für die Existenz von \hyperref[def:riemannscheMetrik]{Riemannschen Metriken}). 
    \item Die Zahl $ n $ heißt \term{Dimension}\label{def:dimension} von $ M $ (eindeutig, wenn $ M $ \hyperref[def:zusammenhaengend]{zusammenhängend} ist, siehe \hyperref[th:satzGebietstreue]{Satz von Gebietstreue}). 
  \end{itemize}
\end{definition}

\begin{example}[{{\hyperref[def:topologischeMannigfaltigkeit]{Topologische Mannigfaltigkeiten}}}]
  \
  \begin{enumerate}
    \setcounter{enumi}{-1}

    \item Eine abzählbare Menge mit \hyperref[bsp:diskreteTopologie]{diskreter Topologie} (jeder Punkt ist offen) ist eine $ 0 $-dimensionale Mannigfaltigkeit.

    \item $ S^1 $ ist eine kompakte, zusammenhängenge $ 1 $-dimensionale Mannigfaltigkeit. \\
      $ \R $ ist \hyperref[def:kompakt]{nichtkompakte}, \hyperref[def:zusammenhaengend]{zusammenhängende} $ 1 $-Mannigfaltigkeit.

    \item Jede offene Teilmenge einer Mannigfaltigkeit ist wieder eine Mannigfaltigkeit, z.B. ist jede offene Teilmenge von $ \R^n $ eine $ n $-dimensionale Mannigfaltigkeit (hier ist Karte = Einschränkung der Identität). \\
    \emph{Spezialfall}: $ \text{GL}(n,\R) = \{ A \in \R^{n \times n} : \det A \neq 0 \} $ ist offene Teilmenge von $ \R^{n^2} $, also eine $ n^2 $-dimensionale Mannigfaltigkeit, denn:
    \begin{itemize}
      \item $ \det : \R^{n \times n} \to \R $ ist stetig
      \item $ \{ 0 \} $ ist abgeschlossen in $ \R $ 
      \item $ \det^{-1} \{ 0 \} $ ist abgeschlossen in $ \R^{n \times n} $
      \item $ \R^{n \times n} \setminus \det^{-1} \{ 0 \} = \text{GL}(n,\R ) $ ist offen in $ \R^{n \times n} $
    \end{itemize}

    \item Die $ n $-dimensionale Sphäre mit Radius $ R > 0 $,
    \begin{equation*}
      S^n_R = \{ x \in \R^{n-1} : \Vert x \Vert = R \}\text{,}
    \end{equation*}
    ist $ n $-dimensionale topologische Mannigfaltigkeit.
    \begin{proof}
      Sei $ (x_1, \dots, x_{n+1}) = p \in S^n_R $, oBdA $ x_{n+1} > 0 $. Man betrachte die Abbildung
      \begin{align*}
        \phi^{-1} : D^n_R \coloneqq \left\{ x \in \R^n : \Vert x \Vert < R \right\} &\to \phi(D^n_R) \subset S^n_R \\
          (x_1, \dots, x_n) &\mapsto \left(x_1, \dots, x_n, \sqrt{R^2-(x_1^2 + \cdots + x_n^2)}\right)
      \end{align*}
      d.h. $ \phi $ ist Einschränkung der Orthogonalprojektion
      \begin{align*}
        \R^{n+1} &\to \R^n \subset \R^{n+1} \\
          (x_1, \dots, x_{n+1}) &\mapsto (x_1, \dots, x_n, 0)
      \end{align*}
      auf $ S_R^n $.
    \end{proof}
  \end{enumerate}
\end{example}